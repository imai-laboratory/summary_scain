\documentclass[11pt]{amsart}
\usepackage{geometry}                % See geometry.pdf to learn the layout options. There are lots.
\geometry{a3paper}                   % ... or a4paper or a5paper or ... 
%\geometry{landscape}                % Activate for for rotated page geometry
%\usepackage[parfill]{parskip}    % Activate to begin paragraphs with an empty line rather than an indent
\usepackage{graphicx}
\usepackage{amssymb}
\usepackage{epstopdf}
\usepackage{xcolor}
\usepackage{udline}
\DeclareGraphicsRule{.tif}{png}{.png}{`convert #1 `dirname #1`/`basename #1 .tif`.png}
\renewcommand{\baselinestretch}{1.5}
\newenvironment{hangall}[1]{\hangindent = 2.5zw\everypar{\hangindent = 2.5zw}}{}

\title{}
\author{}
%\date{}                                           % Activate to display a given date or no date

\begin{document}
\maketitle
%\section{}
%\subsection{}  
\begin{hangall}{}%
A: こんにちわ。暫くでしたが、どうされてましたか?

B: お久しぶりです。わたしは最近、占いに興味を持ち始めました。

A: ああ、占いですか。面白そうですけど、当たる物なんでしょうか。

B: そうですね。私はA型の12月、しかも大晦日生まれなんですが、両方の占いを総合してみるとよく当たっています。

A: 大晦日の生まれですか。じゃあ正月に被って誕生日をないがしろにされたんじゃないでしょうか、って言うのは私の弟が元日産れなんで、誕生日を祝ってもらえなかったってこぼしてましたよ。

B: まさに、弟さんと同じような感じですよ

A: 私は会社を辞めて整体の仕事に就いています。元々柔道をやっていたんでこの仕事は向いているそうなんですよ。

B: \ul{整体師さんなんですね。私は肩がこるので、時々整体に行っています。}\end{hangall}
\end{document}  