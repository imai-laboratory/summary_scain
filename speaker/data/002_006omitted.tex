\documentclass[11pt]{amsart}
\usepackage{geometry}                % See geometry.pdf to learn the layout options. There are lots.
\geometry{a3paper}                   % ... or a4paper or a5paper or ... 
%\geometry{landscape}                % Activate for for rotated page geometry
%\usepackage[parfill]{parskip}    % Activate to begin paragraphs with an empty line rather than an indent
\usepackage{graphicx}
\usepackage{amssymb}
\usepackage{epstopdf}
\usepackage{xcolor}
\usepackage{udline}
\DeclareGraphicsRule{.tif}{png}{.png}{`convert #1 `dirname #1`/`basename #1 .tif`.png}
\renewcommand{\baselinestretch}{1.5}
\newenvironment{hangall}[1]{\hangindent = 2.5zw\everypar{\hangindent = 2.5zw}}{}

\title{}
\author{}
%\date{}                                           % Activate to display a given date or no date

\begin{document}
\maketitle
%\section{}
%\subsection{}  
\begin{hangall}{}%
A: ねえ、アニメって見る?君の名はってやつ、見た?

B: 見た見た!すごく面白かったよね!あなたもアニメ好きなの?

A: 好き!アニメも好きだし、見終わった後、ストーリーの中で使われた場所に実際に出掛けていくのも好きなんだ!

B: それすごく素敵!聖地巡礼てやつね!わたしもアニメ大好きなんだけど周りの人からは夢見がちだって言われるからあまり大きな声で言えなかったの。

A: たしかに夢見がちかもね!でもそれでいいじゃん!わたしなんて、普段、肉体労働してるんだし、頭の中ぐらい夢を見させてほしいわ!

B: こう見えてわたしもバイクに乗るのよ!あなたは鳥取砂丘に行ったことある?

A: \ul{行ったこと無い!あそこって、なにかアニメで使われてたっけ?}\end{hangall}
\end{document}  