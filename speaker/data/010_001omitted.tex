\documentclass[11pt]{amsart}
\usepackage{geometry}                % See geometry.pdf to learn the layout options. There are lots.
\geometry{a3paper}                   % ... or a4paper or a5paper or ... 
%\geometry{landscape}                % Activate for for rotated page geometry
%\usepackage[parfill]{parskip}    % Activate to begin paragraphs with an empty line rather than an indent
\usepackage{graphicx}
\usepackage{amssymb}
\usepackage{epstopdf}
\usepackage{xcolor}
\usepackage{udline}
\DeclareGraphicsRule{.tif}{png}{.png}{`convert #1 `dirname #1`/`basename #1 .tif`.png}
\renewcommand{\baselinestretch}{1.5}
\newenvironment{hangall}[1]{\hangindent = 2.5zw\everypar{\hangindent = 2.5zw}}{}

\title{}
\author{}
%\date{}                                           % Activate to display a given date or no date

\begin{document}
\maketitle
%\section{}
%\subsection{}  
\begin{hangall}{}%
A: こんばんは!今夜はもう、夕飯は済ませましたか?

B: あ、こんばんわ。まだなんですよ。仕事の前には何か食べておかないといけないんですけど、つい忙しくって。食べ物を作る仕事ですから。

A: おや、そんなに忙しいお仕事なの?何の仕事をしてるの?

B: いわゆるパティシエです。このシーズンは結構忙しくって。あ、ごめんなさい、あなたはお仕事は何でしたっけ?

A: わたしはバスの運転手してます。いまから自慢のお手製焼きそばを食おうと思ってたところなんですよ。あなたは、パティシエってことは甘党ってこと?

B: ああ、それは大変なお仕事ですね。私ですか? いえ、大のお酒好きです。占いを信じるたちなんで、今日は何色のお酒にしようか、とか何をツマミにしようかとか。

A: ほほう、占いとは面白いですね!占いって、特に女性が大好きですよねー。

B: それはイケると思いますよ。女性のいる席で盛り上がれるか占ってみましょうか?

A: あ!それ名案です。一緒に暮らしている弟に合コンを主催してもらいますので、是非、一緒に参加してください!

B: ええ是非! 私同様ネガティブな友達で良ければ連れて行きます!

A: \ul{いいですねー、クセのある人がいたほうが、盛り上がります!楽しみにしてます!}\end{hangall}
\end{document}  