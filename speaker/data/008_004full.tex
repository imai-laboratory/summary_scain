\documentclass[11pt]{amsart}
\usepackage{geometry}                % See geometry.pdf to learn the layout options. There are lots.
\geometry{a3paper}                   % ... or a4paper or a5paper or ... 
%\geometry{landscape}                % Activate for for rotated page geometry
%\usepackage[parfill]{parskip}    % Activate to begin paragraphs with an empty line rather than an indent
\usepackage{graphicx}
\usepackage{amssymb}
\usepackage{epstopdf}
\usepackage{xcolor}
\usepackage{udline}
\DeclareGraphicsRule{.tif}{png}{.png}{`convert #1 `dirname #1`/`basename #1 .tif`.png}
\renewcommand{\baselinestretch}{1.5}
\newenvironment{hangall}[1]{\hangindent = 2.5zw\everypar{\hangindent = 2.5zw}}{}

\title{}
\author{}
%\date{}                                           % Activate to display a given date or no date

\begin{document}
\maketitle
%\section{}
%\subsection{}  
\begin{hangall}{}%
A: こんにちは!寒いですねー!

B: ですねー!まあ、新潟生まれ、新潟育ちのわたしからすれば、このぐらいへっちゃらですよ!あなたは、寒いのはお嫌いですか?

A: 嫌いではないです!なんてったって、冬生まれ!誕生日は正月ですから!

B: まあ、ダブルでおめでたいんですね!そしたら、お正月のおせちのあとに、ケーキを食べるの?

A: そうです!よくわかりましたね!でも、あれ、結構お腹いっぱいになるんですよね。正月は、本当にずっと食べてる感じですよ。兄は何もイベントがない6月生まれなので、羨ましいです。

B: お兄様がいらっしゃるんですか。わたしには弟がいて、いつも上の兄弟が欲しかったから、羨ましいわ!

A: そうなんですか?兄貴って、いつもえらそうで面倒ですけどね。ちなみに、わたしは外国生まれですが、兄は日本で生まれたらしいです。本当に兄弟なのか、疑ってたりして。

B: まあ、なんてことを!顔立ちとかは似てないんですか?うちなんて、わたしも弟も、短足なところがソックリだから、疑いようもなく兄弟なのよ。

A: 変なところだけど、似てるんですね!わたしのところは、わたしが見ての通り彫りの深い顔立ち、兄はうすーい顔してます。

B: なるほどー。でもきっとどこかに共通項があるはず!そうだ!今度、わたしがやってる実家の整体院に、ご兄弟で来てくださいな。似てるところを見つけてあげる!

A: あはは!それは楽しそうですね!今度、兄貴を内緒で誘ってみますよ。肩が凝って大変だって言ってたから、ちょうどいいや。

B: まあそうだったの。それじゃ、早めにいらしてくださいね!

A: \ul{了解です!}\end{hangall}
\end{document}  