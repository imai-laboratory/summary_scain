\documentclass[11pt]{amsart}
\usepackage{geometry}                % See geometry.pdf to learn the layout options. There are lots.
\geometry{a3paper}                   % ... or a4paper or a5paper or ... 
%\geometry{landscape}                % Activate for for rotated page geometry
%\usepackage[parfill]{parskip}    % Activate to begin paragraphs with an empty line rather than an indent
\usepackage{graphicx}
\usepackage{amssymb}
\usepackage{epstopdf}
\usepackage{xcolor}
\usepackage{udline}
\DeclareGraphicsRule{.tif}{png}{.png}{`convert #1 `dirname #1`/`basename #1 .tif`.png}
\renewcommand{\baselinestretch}{1.5}
\newenvironment{hangall}[1]{\hangindent = 2.5zw\everypar{\hangindent = 2.5zw}}{}

\title{}
\author{}
%\date{}                                           % Activate to display a given date or no date

\begin{document}
\maketitle
%\section{}
%\subsection{}  
\begin{hangall}{}%
A: こんにちは。田舎暮らしが長いけど、今年は暖かくていい気候だね。

B: ほんとだね。とか言って、インドア派のわたしは、いつでも快適な家の中でばかり、過ごしてるんだけどね!

A: そうなんだ。わたしもね、家の中で過ごすのは好きよ。でも、狭い場所が苦手だから、家は庭付きの吹き抜けのある家に住んでるんだけど。

B: 素敵!そういう家で、音楽でもかけながら、ソーイングできたら幸せなんだろうなあ。

A: ソーイングが好きなのー?わたしはガーデニングが趣味よ。

B: ガーデニングも楽しそう!そうそう、この前さ、いつもみたいにミシンかけてたんだけど、、、

A: うんうん、どうしたの?

B: どうやっても、上糸がつっぱっちゃうんだよねー。でも、機械に疎いから、なにがどうなってるのか、さっぱりで。どうしよー、ミシンが使い物にならない!

A: そんなの、叩いておけばなんとかなるって!って、それは昔のテレビか。わたし、時々適当すぎるのよね。みんなは大らかって言ってくれるけど。

B: アハハ!叩く!なんか、子供の頃に転校した先の先生もそんなこと言ってたなー。

A: そうそう、昔はね、叩けば電化製品なんて直ると思ってたのよ。先生もO型かしら?わたしもO型の大らかタイプなんだけど。

B: いいよ、その性格。わたしは好き。わたしもO型って言いたいけど、わたしの場合、血液型が分からないから残念!

A: そうなのー?あなたはきっと、まじめなA型かしらね。

B: \ul{そうかもね!}\end{hangall}
\end{document}  