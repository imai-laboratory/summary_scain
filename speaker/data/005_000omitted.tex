\documentclass[11pt]{amsart}
\usepackage{geometry}                % See geometry.pdf to learn the layout options. There are lots.
\geometry{a3paper}                   % ... or a4paper or a5paper or ... 
%\geometry{landscape}                % Activate for for rotated page geometry
%\usepackage[parfill]{parskip}    % Activate to begin paragraphs with an empty line rather than an indent
\usepackage{graphicx}
\usepackage{amssymb}
\usepackage{epstopdf}
\usepackage{xcolor}
\usepackage{udline}
\DeclareGraphicsRule{.tif}{png}{.png}{`convert #1 `dirname #1`/`basename #1 .tif`.png}
\renewcommand{\baselinestretch}{1.5}
\newenvironment{hangall}[1]{\hangindent = 2.5zw\everypar{\hangindent = 2.5zw}}{}

\title{}
\author{}
%\date{}                                           % Activate to display a given date or no date

\begin{document}
\maketitle
%\section{}
%\subsection{}  
\begin{hangall}{}%
A: こんにちは。わたし、小学校でチビ!チビ!って、いじめられてるんだ。

B: ええっ!それは傷つくね。大丈夫?

A: うん。大丈夫!いつか見返してやるって思って、頑張ってるから。あなたはどんな人?

B: えらいね!わたしは、仕事を定年してノンビリ暮らしてる老人の一人だよ。毎日釣りばっかりしてるねえ。釣りには興味あるかい?

A: 釣りって、魚を釣るんだよね?あれって、楽しいの?

B: うーん、たしかにそうなんだけど、一人ぼっちって響きは好きじゃないな。一匹狼って言ったほうが良いんじゃないかな?

A: \ul{一匹狼かぁ!それ、かっこいいね!わたしも今度から、そう言おう!ところで、わたし、ニンジンが嫌いなんだけど、あなたは嫌いなものってある?}\end{hangall}
\end{document}  