\documentclass[11pt]{amsart}
\usepackage{geometry}                % See geometry.pdf to learn the layout options. There are lots.
\geometry{a3paper}                   % ... or a4paper or a5paper or ... 
%\geometry{landscape}                % Activate for for rotated page geometry
%\usepackage[parfill]{parskip}    % Activate to begin paragraphs with an empty line rather than an indent
\usepackage{graphicx}
\usepackage{amssymb}
\usepackage{epstopdf}
\usepackage{xcolor}
\usepackage{udline}
\DeclareGraphicsRule{.tif}{png}{.png}{`convert #1 `dirname #1`/`basename #1 .tif`.png}
\renewcommand{\baselinestretch}{1.5}
\newenvironment{hangall}[1]{\hangindent = 2.5zw\everypar{\hangindent = 2.5zw}}{}

\title{}
\author{}
%\date{}                                           % Activate to display a given date or no date

\begin{document}
\maketitle
%\section{}
%\subsection{}  
\begin{hangall}{}%
A: こんにちは、寒いですね。わたし、寒いのが苦手なんですよ。

B: へえ、そうなんだ。南国生まれなの?

A: 南国って言えば南国かな?静岡の南の方です。こんな寒い日は、緑茶が一番!あ、緑茶ありますけど、飲みます?

B: あ、いいですね、緑茶。でも、緑茶よりも、昆布茶が好きかな。わたしは、根っからの大阪人なんで。

A: そうなんですか!わたしは緑茶をよく飲んでるんですけど、大阪の方は昆布茶が好きなんですね。

B: 関西は昆布出汁の文化ですからね。ついでにいえば、たこ焼き文化もあります!一家に一台、たこ焼き器!

A: あ、それ、テレビで見たことあります!本当にあるんですね。そういえば、大阪出身の患者さんもたこ焼き作るのが得意って言ってたなぁ。

B: もちろんいいですよ。物を覚えるのが得意で、1台のたこ焼き器で色んな味のを同時に作れるから、何人来てもOKです!

A: すごいですね!それは楽しみです!じゃぁ患者さんの方にも声をかけておきますね。

B: ぜひぜひ!楽しみにしています!

A: \ul{よろしくお願いします!}\end{hangall}
\end{document}  