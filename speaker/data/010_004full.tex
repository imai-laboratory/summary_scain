\documentclass[11pt]{amsart}
\usepackage{geometry}                % See geometry.pdf to learn the layout options. There are lots.
\geometry{a3paper}                   % ... or a4paper or a5paper or ... 
%\geometry{landscape}                % Activate for for rotated page geometry
%\usepackage[parfill]{parskip}    % Activate to begin paragraphs with an empty line rather than an indent
\usepackage{graphicx}
\usepackage{amssymb}
\usepackage{epstopdf}
\usepackage{xcolor}
\usepackage{udline}
\DeclareGraphicsRule{.tif}{png}{.png}{`convert #1 `dirname #1`/`basename #1 .tif`.png}
\renewcommand{\baselinestretch}{1.5}
\newenvironment{hangall}[1]{\hangindent = 2.5zw\everypar{\hangindent = 2.5zw}}{}

\title{}
\author{}
%\date{}                                           % Activate to display a given date or no date

\begin{document}
\maketitle
%\section{}
%\subsection{}  
\begin{hangall}{}%
A: こんにちは。私は最近、猫を飼い始めました。

B: わあ、羨ましいなあ。私は猫の毛アレルギーだってお母さんが言っていたから飼えないです。飼いたいなあ。猫ちゃん大好き。

A: 猫、かわいいですよ。ふわふわで、もうたまらないです。

B: いいなあ。私のクラスの子達も、飼っている子が大勢なんで、何だか話題に入っていけないなあ。ねえねえ、好きな食べ物って何ですか>

A: アレルギーなら、ちょっと我慢するしかないよね。好きな食べ物は、12月生まれだからケーキ。

B: え、12月生まれはケーキ好きなんだあ。私も12月だからケーキが好きなのかな。この間ね、お母さんがキャロットケーキってのを作って食べさせようとしたんだけど、私ニンジンが嫌いだから。

A: クリスマス大好きだからね。でも、大晦日生まれ。人参は私も苦手だよ。

B: ホッとしたあ。大人だって嫌いなものありますよね。私身長が低くて並ぶ時も一番前なんですけど、好き嫌いがあってもお姉さんみたいに大きくなれますよね?

A: 好き嫌いは、段々なくなるから大丈夫!ちなみに占いにハマってるんだけど、同じ12月なら似た運勢かもね。血液型は私Aだけど、どう?

B: 私はAB型ですっ。私わたし占い師になりたいんです。学習雑誌の付録にオシャレな占いブックが付いてて、それで。

A: 同じ趣味で嬉しいよ。今度、AB型の占いも一緒にみてあげるね。

B: はあい! また占いのお話聞かせて下さい。あと猫ちゃんもさわりたいです。

A: \ul{ちょっとなら、きっと大丈夫だと思うよ。}\end{hangall}
\end{document}  