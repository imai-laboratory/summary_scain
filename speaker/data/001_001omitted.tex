\documentclass[11pt]{amsart}
\usepackage{geometry}                % See geometry.pdf to learn the layout options. There are lots.
\geometry{a3paper}                   % ... or a4paper or a5paper or ... 
%\geometry{landscape}                % Activate for for rotated page geometry
%\usepackage[parfill]{parskip}    % Activate to begin paragraphs with an empty line rather than an indent
\usepackage{graphicx}
\usepackage{amssymb}
\usepackage{epstopdf}
\usepackage{xcolor}
\usepackage{udline}
\DeclareGraphicsRule{.tif}{png}{.png}{`convert #1 `dirname #1`/`basename #1 .tif`.png}
\renewcommand{\baselinestretch}{1.5}
\newenvironment{hangall}[1]{\hangindent = 2.5zw\everypar{\hangindent = 2.5zw}}{}

\title{}
\author{}
%\date{}                                           % Activate to display a given date or no date

\begin{document}
\maketitle
%\section{}
%\subsection{}  
\begin{hangall}{}%
A: こんにちは!つかぬことを質問しますが、人としゃべる時って、あなたはどこに視線をやりますか?

B: わたしの尊敬する父は人の目をみて話せと言っていたのでわたしも目を見て話すようにしています。あなたは違うんですか?

A: お父様の教えは素敵ですね。わたしは、口元を見て話すんですよ。わたしの周りの人もそういう人が多いのか、唇がセクシーねってよく言われます。

B: セクシーだから見ながら話す?へー面白い意見ですね。わたしは外資系企業に勤めているのですが、社員に外国人もいて、彼らってすごく目を見て話すんですよね。

A: ああ、そう言われるとそうかも知れませんね。なんて、知ったかぶりしてみました!わたしは、基本的に日本人相手の仕事なので。

B: うーん保険はすでに入ってるものがあるんですよね。すみません。そういえば、わたしが住んでるアパートにもよく保険外交員がきているような気がするな。

A: そうでしたか!ところで、あなた、好きな食べ物とかありますか?

B: \ul{わたしはカツ丼が好きです。わたしにとってはお袋の味でもあるんですよね。あなたは?}\end{hangall}
\end{document}  