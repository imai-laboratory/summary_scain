\documentclass[11pt]{amsart}
\usepackage{geometry}                % See geometry.pdf to learn the layout options. There are lots.
\geometry{a3paper}                   % ... or a4paper or a5paper or ... 
%\geometry{landscape}                % Activate for for rotated page geometry
%\usepackage[parfill]{parskip}    % Activate to begin paragraphs with an empty line rather than an indent
\usepackage{graphicx}
\usepackage{amssymb}
\usepackage{epstopdf}
\usepackage{xcolor}
\usepackage{udline}
\DeclareGraphicsRule{.tif}{png}{.png}{`convert #1 `dirname #1`/`basename #1 .tif`.png}
\renewcommand{\baselinestretch}{1.5}
\newenvironment{hangall}[1]{\hangindent = 2.5zw\everypar{\hangindent = 2.5zw}}{}

\title{}
\author{}
%\date{}                                           % Activate to display a given date or no date

\begin{document}
\maketitle
%\section{}
%\subsection{}  
\begin{hangall}{}%
A: こんにちは!わたし、以前は高知の大学に通っていたんですが、他にやりたいことがあって、自主退学したんです。

B: こんにちは。やりたいことって何ですか?

A: 好きなボルダリングです!それで、大会に出られるような選手になれるように、集中して取り組もうと思って。

B: すごーい!チャレンジャーですね。そういう思い切りみたいなのが、わたしにはないところで、魅力的です。

A: いやいや、小さい頃から体が柔らかくてボルダリングはやってたんですけどね。でも、夢が捨てきれなかったんです。

B: 夢を追って大学をやめるなんて、勇気がいると思います。わたしなんて、思い切りが足りなくて、ネットオークションになかなか手を出せないでいるんですよ。

A: そうなんですか?でも、ネットオークションって、ちょっと難しい気がしちゃいますもんね。わかるな。

B: でしょ?しかも空気を読むのが苦手なので、いくらぐらいの値をつければいいのか、全然見当もつかないんですよ。

A: そうなんですか。それは羨ましい!わたしも、筋トレグッズのために挑戦してみようかなぁ。

B: \ul{ええ、是非やってみて!それで、コツとかをわたしに教えてください!}\end{hangall}
\end{document}  