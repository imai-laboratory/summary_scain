\documentclass[11pt]{amsart}
\usepackage{geometry}                % See geometry.pdf to learn the layout options. There are lots.
\geometry{a3paper}                   % ... or a4paper or a5paper or ... 
%\geometry{landscape}                % Activate for for rotated page geometry
%\usepackage[parfill]{parskip}    % Activate to begin paragraphs with an empty line rather than an indent
\usepackage{graphicx}
\usepackage{amssymb}
\usepackage{epstopdf}
\usepackage{xcolor}
\usepackage{udline}
\DeclareGraphicsRule{.tif}{png}{.png}{`convert #1 `dirname #1`/`basename #1 .tif`.png}
\renewcommand{\baselinestretch}{1.5}
\newenvironment{hangall}[1]{\hangindent = 2.5zw\everypar{\hangindent = 2.5zw}}{}

\title{}
\author{}
%\date{}                                           % Activate to display a given date or no date

\begin{document}
\maketitle
%\section{}
%\subsection{}  
\begin{hangall}{}%
A: こんにちは。最近はどんなことして過ごしていましたか?

B: 実は最近会社を興しましてね。探偵業をやってるんです。

A: 探偵?それは珍しいお仕事を、しかもご自分で興されたなんて!なかなか大変なお仕事じゃありませんか?

B: ははは。探偵と言っても多い仕事は迷子になったペット探しなんですよ。わたしも犬を飼っているので他人事とは思えなくて。

A: 犬の気持ちが分かって、すぐに発見してもらえそうで心強いですよ!私もウサギを飼っているのですが、万が一迷子になった時はぜひ一緒に探してください。

B: はい、お任せください!そういえば、あなたはどんなお仕事をされていましたっけ?

A: 家政婦を長くやっています。昔から家事が大好きでして。

B: 家政婦ですか!家政婦というとどうしてもドラマのようなイメージをしてしまいますが、実際はいかがですか?

A: 家政婦は見た…ですね!探偵気分を味わえそうなんですけど、実際はなかなかそういった機会にはめぐまれません。

B: そうなんですね!ドラマと現実が違うという点では家政婦も探偵も同じですね。

A: うふふ、ほんとうに。でも人の役に立てるという意味ではやりがいのある仕事ですし、お互い頑張りましょうね。

B: \ul{そうですね!がんばりましょう!}\end{hangall}
\end{document}  