\documentclass[11pt]{amsart}
\usepackage{geometry}                % See geometry.pdf to learn the layout options. There are lots.
\geometry{a3paper}                   % ... or a4paper or a5paper or ... 
%\geometry{landscape}                % Activate for for rotated page geometry
%\usepackage[parfill]{parskip}    % Activate to begin paragraphs with an empty line rather than an indent
\usepackage{graphicx}
\usepackage{amssymb}
\usepackage{epstopdf}
\usepackage{xcolor}
\usepackage{udline}
\DeclareGraphicsRule{.tif}{png}{.png}{`convert #1 `dirname #1`/`basename #1 .tif`.png}
\renewcommand{\baselinestretch}{1.5}
\newenvironment{hangall}[1]{\hangindent = 2.5zw\everypar{\hangindent = 2.5zw}}{}

\title{}
\author{}
%\date{}                                           % Activate to display a given date or no date

\begin{document}
\maketitle
%\section{}
%\subsection{}  
\begin{hangall}{}%
A: あなたは二重で大きな目をしてますよね。

B: そうですかねー。目はそうかもしれませんが、鼻が低いので、全体としては不細工でしょう?

A: そんなことないですよ。わたしは切れ長の目ですが自慢に思ってますし。考えようですよ。

B: いいですね!そういう考え方は見習いたいのですが、自意識過剰なので、みんながわたしの鼻に注目してる気がしてならないんです。

A: そんなに気にすることないと思いますけどね。わたしは証券マンで営業をしていますがあまりお世辞は言わないんですよ。

B: そうなんですか!営業してるのにお世辞を言わず、正直でいられるってすごいですね。商社勤めのわたしも見習いたいと思います。

A: ある意味わたしのスタイルですね。商社だとお忙しいでしょう。休みのひはなにをしてらっしゃるんですか?

B: 休みの日は、家でノンビリとテレビを見たりしてます。あ、唯一、給料日のあとの休日は違いますけど!

A: \ul{家で過ごされる派なんですね。わたしはジョギングをするのが趣味なんです。給料日の後の休日、気になるなあ。}\end{hangall}
\end{document}  