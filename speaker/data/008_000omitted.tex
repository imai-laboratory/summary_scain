\documentclass[11pt]{amsart}
\usepackage{geometry}                % See geometry.pdf to learn the layout options. There are lots.
\geometry{a3paper}                   % ... or a4paper or a5paper or ... 
%\geometry{landscape}                % Activate for for rotated page geometry
%\usepackage[parfill]{parskip}    % Activate to begin paragraphs with an empty line rather than an indent
\usepackage{graphicx}
\usepackage{amssymb}
\usepackage{epstopdf}
\usepackage{xcolor}
\usepackage{udline}
\DeclareGraphicsRule{.tif}{png}{.png}{`convert #1 `dirname #1`/`basename #1 .tif`.png}
\renewcommand{\baselinestretch}{1.5}
\newenvironment{hangall}[1]{\hangindent = 2.5zw\everypar{\hangindent = 2.5zw}}{}

\title{}
\author{}
%\date{}                                           % Activate to display a given date or no date

\begin{document}
\maketitle
%\section{}
%\subsection{}  
\begin{hangall}{}%
A: こんにちは。髪の毛ってさ、どのぐらいのサイクルで切ってる?

B: わたしは姿勢が悪いしあまり見た目に気を使わないので半年ぐらいごとにかな。あなたは?

A: 半年に1回で大丈夫なの?わたしは月1回は切りに行かないと駄目なんだ。髪の毛のボリュームがあるから、早めに切らないと、すぐ爆発しちゃうの!

B: そうなんだ。大変だね。そうだ今度の休み何してる?

A: 休み?いまね、実は休職してるから、毎日が休みなんだよね。だから、毎日普通に過ごしてる感じ。

B: そうなんだ!わたし餃子作りが得意だから今度の休みうちで餃子パーティーなんてどう?

A: わたしは今は北陸だよ。でも毎日暇してるから、関東まで行けるよ!

B: そうだったんだ!?遠いのにありがとう!でも久しぶりにあなたに会えるの楽しみにしてる!

A: 私も楽しみにしてる!今度、時間ができたら、北陸にもおいでよ!こっちはお寿司が美味しいんだ。

B: \ul{行く行く!}\end{hangall}
\end{document}  