\documentclass[11pt]{amsart}
\usepackage{geometry}                % See geometry.pdf to learn the layout options. There are lots.
\geometry{a3paper}                   % ... or a4paper or a5paper or ... 
%\geometry{landscape}                % Activate for for rotated page geometry
%\usepackage[parfill]{parskip}    % Activate to begin paragraphs with an empty line rather than an indent
\usepackage{graphicx}
\usepackage{amssymb}
\usepackage{epstopdf}
\usepackage{xcolor}
\usepackage{udline}
\DeclareGraphicsRule{.tif}{png}{.png}{`convert #1 `dirname #1`/`basename #1 .tif`.png}
\renewcommand{\baselinestretch}{1.5}
\newenvironment{hangall}[1]{\hangindent = 2.5zw\everypar{\hangindent = 2.5zw}}{}

\title{}
\author{}
%\date{}                                           % Activate to display a given date or no date

\begin{document}
\maketitle
%\section{}
%\subsection{}  
\begin{hangall}{}%
A: はあ。わたしなで肩がコンプレックスでさ。あなたはある?コンプレックスとか。

B: そうだなあ、コンプレックスとまでは言わないけど、子供っぽい顔があんまり好きじゃないよ。

A: 歳を取らないってことだしむしろ羨ましいけどな!

B: そうかなあ。でも、化粧品メーカーで、こんなわたしでも大人っぽく見えるメイク用品を開発してるの!

A: それは素敵ね!同じ悩みを持ってる人もいるでしょうし!わたしは尊敬する松下幸之助に憧れて大手電機メーカーに勤めてるのよ。

B: へえ、憧れてる人と同じ業界にいくって、よっぽどだよ。なにか電化製品は開発した?

A: わたしは開発じゃなくて営業なの。あなたはストレス溜まることってない?

B: ストレスはあんまりないよ。ただ、猫背がちょっと気になるかも。

A: 猫背は直そうと思えば直せるから大丈夫だよ。わたしのなで肩は直しようがないからさ。

B: なで肩って、しとやかっぽくて良いと思うよ。着物が似合いそう。

A: \ul{そんな風に言われたのは初めてだなあ。ありがとう!なんだか元気になってきた。}\end{hangall}
\end{document}  