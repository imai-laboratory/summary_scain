\documentclass[11pt]{amsart}
\usepackage{geometry}                % See geometry.pdf to learn the layout options. There are lots.
\geometry{a3paper}                   % ... or a4paper or a5paper or ... 
%\geometry{landscape}                % Activate for for rotated page geometry
%\usepackage[parfill]{parskip}    % Activate to begin paragraphs with an empty line rather than an indent
\usepackage{graphicx}
\usepackage{amssymb}
\usepackage{epstopdf}
\usepackage{xcolor}
\usepackage{udline}
\DeclareGraphicsRule{.tif}{png}{.png}{`convert #1 `dirname #1`/`basename #1 .tif`.png}
\renewcommand{\baselinestretch}{1.5}
\newenvironment{hangall}[1]{\hangindent = 2.5zw\everypar{\hangindent = 2.5zw}}{}

\title{}
\author{}
%\date{}                                           % Activate to display a given date or no date

\begin{document}
\maketitle
%\section{}
%\subsection{}  
\begin{hangall}{}%
A: こんにちは、お元気していましたか?

B: はい、元気にしています。今、看護師を目指して千葉の看護学校に通っています。

A: すごいですね、頑張ってくださいね!わたしは地元埼玉から、引っ越して東京で暮らしています。

B: 東京ですか!時々遊びに行っています。看護師目指しているのに、実は注射苦手なのがネックです。

A: 看護師さんでもやはり注射が苦手な方がいるのですね。わたしは田舎出身の夏生れですが、雷が苦手なたちです。

B: そういうこと、ありますよね。ちなみに趣味がヨガなのですが、何か趣味ありますか?

A: ヨガもやってみたいですね。特にこれと言って趣味はないのですが、地図がちゃんと読めなくて、方向音痴の割に、街歩きが好きなんです。

B: わたしも方向音痴なので、気持ちわかります。

A: \ul{ですよね、なかなか道は覚えられません。お休みの日なんかは何をされているんですか?}\end{hangall}
\end{document}  