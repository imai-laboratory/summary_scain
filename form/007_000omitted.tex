\documentclass[11pt]{amsart}
\usepackage{geometry}                % See geometry.pdf to learn the layout options. There are lots.
\geometry{a3paper}                   % ... or a4paper or a5paper or ... 
%\geometry{landscape}                % Activate for for rotated page geometry
%\usepackage[parfill]{parskip}    % Activate to begin paragraphs with an empty line rather than an indent
\usepackage{graphicx}
\usepackage{amssymb}
\usepackage{epstopdf}
\usepackage{xcolor}
\usepackage{udline}
\DeclareGraphicsRule{.tif}{png}{.png}{`convert #1 `dirname #1`/`basename #1 .tif`.png}
\renewcommand{\baselinestretch}{1.5}
\newenvironment{hangall}[1]{\hangindent = 2.5zw\everypar{\hangindent = 2.5zw}}{}

\title{}
\author{}
%\date{}                                           % Activate to display a given date or no date

\begin{document}
\maketitle
%\section{}
%\subsection{}  
\begin{hangall}{}%
A: こんにちは!もうすぐ土用の丑の日ですね。ウナギはお好きですか?

B: そういえばそうですね。うなぎ好きですよ。あなたも?

A: はい!大好きです。わたしは年中温暖な気候といわれる、浜松の出身なので、ウナギが名物でソウルフードでもあるんですよ。

B: もちろん知ってますよ!だから土用の丑の日を覚えてたんですね。わたしは高地で育ったので温暖な機構のところには憧れがあるんですよね。

A: 高地ですか。それって、寒いだけじゃなくて、空気が薄かったりもしませんか?

B: そうかもしれません。わたしは昔からジョギングが趣味なのですが、そんなに苦しいと思ったことありませんからね。

A: わあ、すごいですね。わたしは弱虫なので、空気が薄いと聞いただけで不安になります。そういう自分が本当は嫌なんです。

B: 何事もやって見たら変わりますよ。日課を変えるとか。何か日課とかありますか?

A: 日課ですかあ。人によく、のんびりしてる性格と言われるほど時間にルーズなので、毎日決まった時間に何かするということはないですねー。

B: じゃあそれを変えていきましょう!わたしは朝にお弁当を作るのが日課なんですが、そういう些細なことから始めてみましょう!

A: \ul{分かりました!頑張ってチャレンジしてみます!}\end{hangall}
\end{document}  