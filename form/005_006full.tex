\documentclass[11pt]{amsart}
\usepackage{geometry}                % See geometry.pdf to learn the layout options. There are lots.
\geometry{a3paper}                   % ... or a4paper or a5paper or ... 
%\geometry{landscape}                % Activate for for rotated page geometry
%\usepackage[parfill]{parskip}    % Activate to begin paragraphs with an empty line rather than an indent
\usepackage{graphicx}
\usepackage{amssymb}
\usepackage{epstopdf}
\usepackage{xcolor}
\usepackage{udline}
\DeclareGraphicsRule{.tif}{png}{.png}{`convert #1 `dirname #1`/`basename #1 .tif`.png}
\renewcommand{\baselinestretch}{1.5}
\newenvironment{hangall}[1]{\hangindent = 2.5zw\everypar{\hangindent = 2.5zw}}{}

\title{}
\author{}
%\date{}                                           % Activate to display a given date or no date

\begin{document}
\maketitle
%\section{}
%\subsection{}  
\begin{hangall}{}%
A: こんにちは。最近は元気ですか?

B: こんにちは、元気ですよ。

A: わたしもです。一年中温暖な地域で育ったせいか、のんびりした性格だってよく人に言われるんです。

B: わたしも離島の生まれ、育ちでのんびりしてるねってよく言われます。

A: そうなんですね!でも、自分ではそんなにのんびりしているだけではなくて、ちょっと弱虫なところもあるかなと思っています。

B: そうなのですね。わたしは離島生まれを馬鹿にされてきたのでそれをバネに勉強して今は商社で働いています。

A: 商社にお勤めなんて、すごいです。わたしは最初に就職してから一度も転職せずに働いているんですが、なかなかキャリアアップできません。

B: なかなかキャリアアップしていくのも難しいですよね。わたしは商談を決めた時のご馳走にステーキを食べます。

A: わたしもです!ウナギが大好物なので、給料日にはお気に入りの店に食べに行くんです。

B: うなぎも美味しいですね。

A: そうなんですよ。ほかに何か趣味はありますか?わたしは無趣味なので、参考にさせていただきたいです。

B: 趣味はないですね。鼻が低いのがコンプレックスです。

A: えーっ、そんなに低くは見えませんよ!とっても素敵だと思います。

B: \ul{ありがとうございます。自信を持っていきます。}\end{hangall}
\end{document}  