\documentclass[11pt]{amsart}
\usepackage{geometry}                % See geometry.pdf to learn the layout options. There are lots.
\geometry{a3paper}                   % ... or a4paper or a5paper or ... 
%\geometry{landscape}                % Activate for for rotated page geometry
%\usepackage[parfill]{parskip}    % Activate to begin paragraphs with an empty line rather than an indent
\usepackage{graphicx}
\usepackage{amssymb}
\usepackage{epstopdf}
\usepackage{xcolor}
\usepackage{udline}
\DeclareGraphicsRule{.tif}{png}{.png}{`convert #1 `dirname #1`/`basename #1 .tif`.png}
\renewcommand{\baselinestretch}{1.5}
\newenvironment{hangall}[1]{\hangindent = 2.5zw\everypar{\hangindent = 2.5zw}}{}

\title{}
\author{}
%\date{}                                           % Activate to display a given date or no date

\begin{document}
\maketitle
%\section{}
%\subsection{}  
\begin{hangall}{}%
A: はじめまして。私は三重県の県庁所在地に住んでいる獣医です。

B: こんにちは。わたしは心理カウンセラーです

A: そうなんですね。動物も心を病んでいる時があるから、いつかアドバイスいただいても良いですか?

B: はい、もちろんどうぞ。

A: ありがとうございます。ちなみに私は閉所恐怖症なのですが、今度相談してもよいですか?

B: はい、そちらもお待ちしていますね。さて、一つ気になっていることがあるんですが。

A: え、なんでしょうか?

B: わたしの故郷は静岡なんですが、だからというわけなのか、よく好きで飲むのは緑茶なんです。あなたは三重県に住んでいらっしゃいますが、やはり伊勢エビとかをよく食べるとか?

A: 静岡はお茶、有名ですよね。三重は、伊勢えびとか松坂牛とか有名ですが、高級で中々食べれないんですよ。

B: そうなんですね。いつも出身地の名産を必ず食べたり飲んだりするイメージを持たれていて気になりまして。地元だからといって、高級なものは無理ですよね。

A: ですね、でも地元の津ぎょうざとかは普通に食べたりしますよ。名物の種類によりますよね。

B: なるほどそうなんですね。津餃子気になります。今度作り方を教えてください。

A: はい、いつかお教えしますね。

B: \ul{楽しみにしていますね。}\end{hangall}
\end{document}  