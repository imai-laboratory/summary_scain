\documentclass[11pt]{amsart}
\usepackage{geometry}                % See geometry.pdf to learn the layout options. There are lots.
\geometry{a3paper}                   % ... or a4paper or a5paper or ... 
%\geometry{landscape}                % Activate for for rotated page geometry
%\usepackage[parfill]{parskip}    % Activate to begin paragraphs with an empty line rather than an indent
\usepackage{graphicx}
\usepackage{amssymb}
\usepackage{epstopdf}
\usepackage{xcolor}
\usepackage{udline}
\DeclareGraphicsRule{.tif}{png}{.png}{`convert #1 `dirname #1`/`basename #1 .tif`.png}
\renewcommand{\baselinestretch}{1.5}
\newenvironment{hangall}[1]{\hangindent = 2.5zw\everypar{\hangindent = 2.5zw}}{}

\title{}
\author{}
%\date{}                                           % Activate to display a given date or no date

\begin{document}
\maketitle
%\section{}
%\subsection{}  
\begin{hangall}{}%
A: あ、どうも。えっと、すみません、なんか。いや、変なこと言ってるかな、わたし。

B: うん?どうしたの?

A: あ、すみません。初対面の人と話すと、どうも緊張しちゃって。しかも、人の目を見て話すのも苦手なんです。怒ってます?

B: 怒ってないですよ。わたし、最近愛媛から転校してきたばかりの中学生なんです。わたしもどちらかと言うと、初対面の人苦手で、まだ友達いないんです。

A: あ、そうなんだ。友達って、いっぱい必要だと思う?

B: うーん、わたしの家は母一人子一人の母子家庭なんだけど、やっぱり周りに知り合いが多い方が、暮らしやすいのかもって思ってるよ。

A: たしかにね。わたしと違って、あなたなら、しっかりしてるし、きっとすぐに友達ができるよ、うん。

B: そうかなぁ?わたしね、よく怖がりって言われて、いじめられるんだよね。暗い場所も苦手だし、ホラー映画も苦手なの。

A: そんなの、みんなそうだよ。わたしなんて、愛知で育ったけど、未だに愛知の田舎の方の街灯がない場所は、夜歩けないもん。

B: 街灯がないの!?それは困った。わたしも、そんなとこ歩けないよ。

A: 同じだね。だから、暗くなってきたら、家でプラモデルを作って過ごすのが、わたしの基本スタイルなんだよ。

B: なるほど!それなら、気がまぎれそうだね!わたしも、気を紛らわせられるもの、探してみよう!

A: うん、それがいいよ。
\end{hangall}
\end{document}  