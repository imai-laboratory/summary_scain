\documentclass[11pt]{amsart}
\usepackage{geometry}                % See geometry.pdf to learn the layout options. There are lots.
\geometry{a3paper}                   % ... or a4paper or a5paper or ... 
%\geometry{landscape}                % Activate for for rotated page geometry
%\usepackage[parfill]{parskip}    % Activate to begin paragraphs with an empty line rather than an indent
\usepackage{graphicx}
\usepackage{amssymb}
\usepackage{epstopdf}
\usepackage{xcolor}
\usepackage{udline}
\DeclareGraphicsRule{.tif}{png}{.png}{`convert #1 `dirname #1`/`basename #1 .tif`.png}
\renewcommand{\baselinestretch}{1.5}
\newenvironment{hangall}[1]{\hangindent = 2.5zw\everypar{\hangindent = 2.5zw}}{}

\title{}
\author{}
%\date{}                                           % Activate to display a given date or no date

\begin{document}
\maketitle
%\section{}
%\subsection{}  
\begin{hangall}{}%
A: こんにちは。わたしは川のそばにある団地に住んでいます。



B: こんにちは!私はスーパー近くのアパート暮らしですよ!



A: スーパー近くって、楽そうで良いですね。わたしも、そういうところに住みたかったなぁ。



B: 便利です!外資系に勤めていることもあり、忙しくて、ついついスーパーの総菜を買ってしまいます。



A: そうですよね。スーパーの総菜って、意外とレベル高くて美味しいですもんね。わたしも鉄道マンで一人暮らしなんで、よくわかります。



B: そうなんですよね!でも大好きな食べ物は、総菜ダメだなと思いました。私にとっておふくろの味のカツ丼を買ってきて食べたんですけど、がっかりしちゃって…。



A: あはは、確かに!おふくろの味には勝てないですもんね!好きな食べ物なら、なおさらですよね。



B: そうなんですよね。ところで、尊敬している人はいますか?私は父親を尊敬しているんです。



A: お父様ですか。きっと素晴らしい人なんですね。わたしは、坂本龍馬を尊敬しています。



B: 確かに、あの行動力は素晴らしい。今の世の中にも坂本龍馬のような人が現れてほしいですね。\end{hangall}
\end{document}  