\documentclass[11pt]{amsart}
\usepackage{geometry}                % See geometry.pdf to learn the layout options. There are lots.
\geometry{a3paper}                   % ... or a4paper or a5paper or ... 
%\geometry{landscape}                % Activate for for rotated page geometry
%\usepackage[parfill]{parskip}    % Activate to begin paragraphs with an empty line rather than an indent
\usepackage{graphicx}
\usepackage{amssymb}
\usepackage{epstopdf}
\usepackage{xcolor}
\usepackage{udline}
\DeclareGraphicsRule{.tif}{png}{.png}{`convert #1 `dirname #1`/`basename #1 .tif`.png}
\renewcommand{\baselinestretch}{1.5}
\newenvironment{hangall}[1]{\hangindent = 2.5zw\everypar{\hangindent = 2.5zw}}{}

\title{}
\author{}
%\date{}                                           % Activate to display a given date or no date

\begin{document}
\maketitle
%\section{}
%\subsection{}  
\begin{hangall}{}%
A: こんにちは。お元気ですか?

B: こんにちは元気ですよ。今日も趣味のジョギングをしてきたところです。あなたは元気?

A: はい、元気にしています。ジョギングとは、健康的ですね!

B: 健康を意識して生活しています。朝にお弁当を作るのが日課です。

A: お弁当作りも大変そうですね。わたしは演劇部に入っていたからか、声が普段から大きくて。そのせいか、お腹がすぐすいちゃいます。

B: なれると案外楽しくなりますよ。演劇部素敵ですね。

A: ありがとうございます。お弁当作り、やってみようかな。やりはじめると熱中するタイプなので、アドバイスをお願いするかもしれません。

B: 任せて下さい!演劇部ということはあなたは学生さんですか?わたしは証券マンなんです。

A: お弁当の件、宜しくお願いいたします。演劇部は、高校くらいまでですね。今は、普通のサラリーマンです。

B: そうだったのですね、わたし学生の頃は高地で育ったのですがあった部活は山岳部くらいでしたよ。

A: なるほどです。私は暑い地域で育ったのですが、暑さを吹き飛ばすかのように練習していました。
\end{hangall}
\end{document}  