\documentclass[11pt]{amsart}
\usepackage{geometry}                % See geometry.pdf to learn the layout options. There are lots.
\geometry{a3paper}                   % ... or a4paper or a5paper or ... 
%\geometry{landscape}                % Activate for for rotated page geometry
%\usepackage[parfill]{parskip}    % Activate to begin paragraphs with an empty line rather than an indent
\usepackage{graphicx}
\usepackage{amssymb}
\usepackage{epstopdf}
\usepackage{xcolor}
\usepackage{udline}
\DeclareGraphicsRule{.tif}{png}{.png}{`convert #1 `dirname #1`/`basename #1 .tif`.png}
\renewcommand{\baselinestretch}{1.5}
\newenvironment{hangall}[1]{\hangindent = 2.5zw\everypar{\hangindent = 2.5zw}}{}

\title{}
\author{}
%\date{}                                           % Activate to display a given date or no date

\begin{document}
\maketitle
%\section{}
%\subsection{}  
\begin{hangall}{}%
A: 最近はわたしは映画鑑賞が趣味で、洋画から邦画まで見るのにハマってるんですが、何か趣味はありますか?

B: わあ、映画も好きですが、わたしは音楽鑑賞です。ショパンが好きですね。シュピルマンの話見ました?戦場のピアニスト?

A: 戦場のピアニストは見ました、感動しますよね!音楽はクラシックを聴かれるんですね。

B: そうなんです。楽器メーカー勤務なので。店でピアノの弾き比べできますよ。スタンウェイからヤマハまで!

A: へえ、すごいですね!音楽に精通してそうですね。わたしの誕生日は11月22日のいい夫婦の日なんですが、その日に夫婦揃ってクラシックコンサートに行くのも素敵かもしれません。

B: え!いつも行かれるんですか?仲が良くって羨ましい!わたしは群馬の田舎で秋に生まれたんですが、夕暮れになるといつも鈴虫が鳴いていて。その影響ですかね。音楽好きは!

A: 秋の自然の音はすごく風流な感じがしますね。田舎だとより音に対して敏感になりそうな気がします。

B: はい、静かなコンサートホールでヴァイオリンのように鈴虫が鳴く感じですね。

A: 素敵ですね!自然の静かな音も格別ですが、わたしはキャビンアテンダントをしていて、喧騒を離れた田舎の晴れている日が好きなんですが、たまに田舎に行くとすがすがしい気持ちにいつもさせられます。

B: スッチーさんなんですね!カッコいい!じゃあ是非パリのサルプレイエルに行ってみて下さい!

A: サルプレイエルはぜひ行ってみたいですね!行ったことあるのですか?

B: はい、ジョシュアのヴァイオリンを聴いたことがあります。
\end{hangall}
\end{document}  