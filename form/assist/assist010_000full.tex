\documentclass[11pt]{amsart}
\usepackage{geometry}                % See geometry.pdf to learn the layout options. There are lots.
\geometry{a3paper}                   % ... or a4paper or a5paper or ... 
%\geometry{landscape}                % Activate for for rotated page geometry
%\usepackage[parfill]{parskip}    % Activate to begin paragraphs with an empty line rather than an indent
\usepackage{graphicx}
\usepackage{amssymb}
\usepackage{epstopdf}
\usepackage{xcolor}
\usepackage{udline}
\DeclareGraphicsRule{.tif}{png}{.png}{`convert #1 `dirname #1`/`basename #1 .tif`.png}
\renewcommand{\baselinestretch}{1.5}
\newenvironment{hangall}[1]{\hangindent = 2.5zw\everypar{\hangindent = 2.5zw}}{}

\title{}
\author{}
%\date{}                                           % Activate to display a given date or no date

\begin{document}
\maketitle
%\section{}
%\subsection{}  
\begin{hangall}{}%
A: お久しぶりです。お元気でしたか?

B: お久しぶりです!最近ホームヘルパーに転職したので、ちょっと疲れ気味です。あなたは?

A: そうなんですね。私はソーシャルワーカーをしているので、似た職種ですね。

B: そうですか、奇遇ですね。趣味は将棋なんですが、趣味まで一緒ではないですよね、

A: 残念ながら、私の趣味は料理をすることなんです。

B: へえ、料理が上手なんて尊敬しますよ。時間ある時は何をしているんですか?

A: 私は将棋できないので、やってみたいです。料理以外は、最近太ってるので苦手な運動を始めました。

B: 運動は体にいいのでやるべきですね。わたしは筋トレが日課なので、胸板が厚いんですよ。

A: 筋トレですか!運動全般が基本的にダメなので、今度教えてもらいたいです。

B: ぜひ、わたしでよければ!山梨の方言が好きなんですが、富士山の側の湖湖畔などで、サイクリングや運動をすがすがしくやりたいですね。

A: なるほど。それもいいですね!屋内での仕事が多いので、外に出てリフレッシュしたいです。

B: ですね、リフレッシュして明日からもお仕事頑張りましょう。
\end{hangall}
\end{document}  