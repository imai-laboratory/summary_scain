\documentclass[11pt]{amsart}
\usepackage{geometry}                % See geometry.pdf to learn the layout options. There are lots.
\geometry{a3paper}                   % ... or a4paper or a5paper or ... 
%\geometry{landscape}                % Activate for for rotated page geometry
%\usepackage[parfill]{parskip}    % Activate to begin paragraphs with an empty line rather than an indent
\usepackage{graphicx}
\usepackage{amssymb}
\usepackage{epstopdf}
\usepackage{xcolor}
\usepackage{udline}
\DeclareGraphicsRule{.tif}{png}{.png}{`convert #1 `dirname #1`/`basename #1 .tif`.png}
\renewcommand{\baselinestretch}{1.5}
\newenvironment{hangall}[1]{\hangindent = 2.5zw\everypar{\hangindent = 2.5zw}}{}

\title{}
\author{}
%\date{}                                           % Activate to display a given date or no date

\begin{document}
\maketitle
%\section{}
%\subsection{}  
\begin{hangall}{}%
A: こんにちは!私は大阪在住で、たこ焼きを焼くのが得意なんですが、何か趣味とかありますか?



B: たこやきおいしいですよね。私は和歌山に住んでいて、ツーリングをよくしますね。



A: 和歌山なら大阪に近いですね。ツーリングは楽しそう。わたしはたこ焼きを食べる以外には、本を読むことが趣味なんです。



B: すごいですね。私は文字が苦手で、仕事も肉体労働をしています。本を読める人は尊敬します。



A: アクティブな方なんですね、わたしは翻訳家をしていて、あまり家から出ないので、活発な方が羨ましいです。



B: そうなんですね。確かに活発で、仕事以外でも、アニメの聖地巡礼とかで色々なところを廻ったりしてます。



A: 多趣味なんですね、わたしは細かい作業や暗記など、ちまちました作業が得意なのか、活動的な人に憧れますね。



B: 器用なんですね。私は活発な反面マイペースだと言われてしまうこともあります。



A: そうですね、家でできる趣味があると、気軽にストレス発散ができそうな気がします。



B: ええ、そうだと思います。\end{hangall}
\end{document}  