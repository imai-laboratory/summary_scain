\documentclass[11pt]{amsart}
\usepackage{geometry}                % See geometry.pdf to learn the layout options. There are lots.
\geometry{a3paper}                   % ... or a4paper or a5paper or ... 
%\geometry{landscape}                % Activate for for rotated page geometry
%\usepackage[parfill]{parskip}    % Activate to begin paragraphs with an empty line rather than an indent
\usepackage{graphicx}
\usepackage{amssymb}
\usepackage{epstopdf}
\usepackage{xcolor}
\usepackage{udline}
\DeclareGraphicsRule{.tif}{png}{.png}{`convert #1 `dirname #1`/`basename #1 .tif`.png}
\renewcommand{\baselinestretch}{1.5}
\newenvironment{hangall}[1]{\hangindent = 2.5zw\everypar{\hangindent = 2.5zw}}{}

\title{}
\author{}
%\date{}                                           % Activate to display a given date or no date

\begin{document}
\maketitle
%\section{}
%\subsection{}  
\begin{hangall}{}%
A: どうも、お久しぶりです。ついこの間、旅行先の阿蘇で乗馬を体験してきました!

B: お久しぶりです。阿蘇に行ってきたんですか!わたしもアウトドアレジャーが趣味なので、乗馬してみたいなぁ。

A: いいですよ、乗馬!わたしはこの1回だけで、大好きになっちゃいました。アウトドアはどんなことをするんですか?

B: 人が多い場所が苦手なもので、どちらかというと少人数で動けるものが好きですね。たとえば、秘境で釣りとか。

A: ああ、それは仕事の息抜きにも良さそうですね。ところで、お仕事はいまは何かされてますか?

B: 仕事はキャリアコンサルタントをしています。仕事の関係で、最近徳島に移住したのですが、徳島は良いところですよ。

A: 徳島ですか。行ったことがないです。こちらは三重の県庁所在地で獣医としてやってるので、移住は夢の夢だなあ。

B: それじゃぁ、地元の患者さんからの信頼も厚いんでしょうね。ところで、ドライブが好きなんですが、車ってよく乗りますか?

A: 乗りますよ!ただ、人より少し足が長めなので、運転席が窮屈ですね!あ、自慢じゃないですよ!

B: あはは!たしかに、足長いですよね。運転に関しては、足が短い私の方が得ですね。まさかメリットになっていたとは!

A: そのポジティブさも、仕事柄でしょうかねえ。お仕事での成功をお祈りしてますよ!

B: ありがとうございます!あなたも、獣医のお仕事でのご成功、お祈りしています。

A: いや、どうも、ありがとう!
\end{hangall}
\end{document}  