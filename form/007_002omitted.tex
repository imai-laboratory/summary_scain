\documentclass[11pt]{amsart}
\usepackage{geometry}                % See geometry.pdf to learn the layout options. There are lots.
\geometry{a3paper}                   % ... or a4paper or a5paper or ... 
%\geometry{landscape}                % Activate for for rotated page geometry
%\usepackage[parfill]{parskip}    % Activate to begin paragraphs with an empty line rather than an indent
\usepackage{graphicx}
\usepackage{amssymb}
\usepackage{epstopdf}
\usepackage{xcolor}
\usepackage{udline}
\DeclareGraphicsRule{.tif}{png}{.png}{`convert #1 `dirname #1`/`basename #1 .tif`.png}
\renewcommand{\baselinestretch}{1.5}
\newenvironment{hangall}[1]{\hangindent = 2.5zw\everypar{\hangindent = 2.5zw}}{}

\title{}
\author{}
%\date{}                                           % Activate to display a given date or no date

\begin{document}
\maketitle
%\section{}
%\subsection{}  
\begin{hangall}{}%
A: こんにちは!ランチで麻婆豆腐を食べたら、豆腐が熱くて舌をヤケドしちゃったわ!

B: そうなんだ。熱いのって、気を付けないとやばいよね。わたしも名古屋めしが好きで、土手鍋食べて、よくやけどする。

A: 土手鍋、美味しそう!食べ歩きが趣味なんだけど、猫舌だから、色々制限があるのよ。でも土手鍋、食べたいー!

B: 土手鍋、美味しいよ。愛知で育ったんだけど、小さい頃から食べてるんだ。

A: うらやましい!愛知って、色々美味しいものがたくさんあるよね?

B: なるほど。どおりで、話しやすいと思った。初対面の人って苦手なはずなのに、あなたは違うから。そんな人もいるんだね。

A: わたしだけじゃないよ!世の中には合う人が沢山いると思うから、全然だいじょうぶ!

B: \ul{ありがとう、そういう人に出会えると良いな。プラモデルが趣味なんだけど、そういう人でプラモデル好きだったら、もっと嬉しいな。}\end{hangall}
\end{document}  