\documentclass[11pt]{amsart}
\usepackage{geometry}                % See geometry.pdf to learn the layout options. There are lots.
\geometry{a3paper}                   % ... or a4paper or a5paper or ... 
%\geometry{landscape}                % Activate for for rotated page geometry
%\usepackage[parfill]{parskip}    % Activate to begin paragraphs with an empty line rather than an indent
\usepackage{graphicx}
\usepackage{amssymb}
\usepackage{epstopdf}
\usepackage{xcolor}
\usepackage{udline}
\DeclareGraphicsRule{.tif}{png}{.png}{`convert #1 `dirname #1`/`basename #1 .tif`.png}
\renewcommand{\baselinestretch}{1.5}
\newenvironment{hangall}[1]{\hangindent = 2.5zw\everypar{\hangindent = 2.5zw}}{}

\title{}
\author{}
%\date{}                                           % Activate to display a given date or no date

\begin{document}
\maketitle
%\section{}
%\subsection{}  
\begin{hangall}{}%
A: こんにちは。わたし、銀行に勤めてるんですけど、最近石川県の方に転勤してきたんですよ。よろしくお願いします。

B: こちらこそ、よろしくお願いします。外から来ると、石川って、方言がきつく聞こえます?

A: そうですか?今のところは、そんなに気になったことないですけど。でも、わたし、断るのが苦手なくらい引っ込み思案なので、きつく言ってもらった方が助かるかも。

B: そうだったんですか。銀行勤めで断るのが苦手だと、業務に支障ないですか?

A: そうなんですよ。頼まれたら引き受けちゃうので、仕事が忙しくって。まぁ、その分評判は良いのかもしれませんけど。

B: なるほど!評判が良いのは羨ましいな。わたしはホームヘルパーをやってるんですが、お年寄りからちょっと怖がられるんですよ。

A: そうなんですか?意外です。どうしてだろう?

B: どうやら、胸板が厚くて、いかつく見えるみたいなんです。初対面では怖がられますが、将棋が得意なので、それでプラマイゼロかな。

A: \ul{ボードゲームっていう意味では似てますよね。趣味がゲームなので、将棋もできますよ!今度一緒に勝負してみたいですね。}\end{hangall}
\end{document}  