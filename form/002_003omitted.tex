\documentclass[11pt]{amsart}
\usepackage{geometry}                % See geometry.pdf to learn the layout options. There are lots.
\geometry{a3paper}                   % ... or a4paper or a5paper or ... 
%\geometry{landscape}                % Activate for for rotated page geometry
%\usepackage[parfill]{parskip}    % Activate to begin paragraphs with an empty line rather than an indent
\usepackage{graphicx}
\usepackage{amssymb}
\usepackage{epstopdf}
\usepackage{xcolor}
\usepackage{udline}
\DeclareGraphicsRule{.tif}{png}{.png}{`convert #1 `dirname #1`/`basename #1 .tif`.png}
\renewcommand{\baselinestretch}{1.5}
\newenvironment{hangall}[1]{\hangindent = 2.5zw\everypar{\hangindent = 2.5zw}}{}

\title{}
\author{}
%\date{}                                           % Activate to display a given date or no date

\begin{document}
\maketitle
%\section{}
%\subsection{}  
\begin{hangall}{}%
A: お疲れ様です!あ、ごめんなさいわたし声大きですよね。

B: はい!ちょっとビックリしちゃいました。わたし、弱虫なんです。こんな自分から早く脱却したいんですけど。

A: わたし演劇部に入っていたのでつい声張っちゃって、うるさいってよく言われるんです。あなたも大声で話してみたら変わるかもしれませんよ?

B: なるほどー。わたしはよく周りから、のんびりしてるねーって言われるんですけど、そういう性格も声の大きさと関係があると思いますか?

A: それは関係ないかもしれません。演劇部にのんびりな子もいましたし。あなたは劇とか見ますか?

B: いいえ、わたしは演劇とか芸術には疎くて。あなたは今は、プロの役者さんなんですか?

A: いいえ。今はただのサラリーマンです。でも演劇は今でも好きでよく見に行くんですけど、わたしすごく涙もろくてすぐ泣いてしまうんですよね。

B: ええ。新卒で就職してからずっと同じところで勤めているので、もう少し冒険をした方が良いのかなあと思っているところです。

A: なるほど。若いからどんどん挑戦するのはいいことだと思いますよ。

B: \ul{そうですよね!大好きなうな丼を食べて気合いを入れて、転職に取り組んでみたいと思います!}\end{hangall}
\end{document}  