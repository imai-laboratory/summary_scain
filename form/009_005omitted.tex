\documentclass[11pt]{amsart}
\usepackage{geometry}                % See geometry.pdf to learn the layout options. There are lots.
\geometry{a3paper}                   % ... or a4paper or a5paper or ... 
%\geometry{landscape}                % Activate for for rotated page geometry
%\usepackage[parfill]{parskip}    % Activate to begin paragraphs with an empty line rather than an indent
\usepackage{graphicx}
\usepackage{amssymb}
\usepackage{epstopdf}
\usepackage{xcolor}
\usepackage{udline}
\DeclareGraphicsRule{.tif}{png}{.png}{`convert #1 `dirname #1`/`basename #1 .tif`.png}
\renewcommand{\baselinestretch}{1.5}
\newenvironment{hangall}[1]{\hangindent = 2.5zw\everypar{\hangindent = 2.5zw}}{}

\title{}
\author{}
%\date{}                                           % Activate to display a given date or no date

\begin{document}
\maketitle
%\section{}
%\subsection{}  
\begin{hangall}{}%
A: こんばんは。お久しぶりですね。今わたしタクシー運転手をしているんですよ。

B: こんばんは!大変そうなお仕事ですね。わたしはパティシエをやってるので、疲れたときは甘いものを買いに来てくださいよ。

A: いいですね。甘いもの。わたしには双子の兄弟がいるのですが、彼も好きなので二人分買いに行きますね。

B: ぜひぜひ!いや、双子だったら、兄弟のほうが来たら、あなたかどうか、見分けがつかなそうだな!

A: はは。それもそうかもしれません。お家はお近くでしたっけ?

B: ええ、そうですよ。でもねえ、ちょっと方角が良くないって占いに出ていたんで、引っ越そうか検討してるところです。

A: そうなんですね。わたしは記憶力に自信がないから、占い師さんに言われたこともすぐに忘れちゃうかも。ダメですよね、ほんとは。

B: いやいや、嫌なことはすぐに忘れて、良いことだけ覚えておけば良いんですよ!

A: \ul{そうですね。都合よくね。}\end{hangall}
\end{document}  