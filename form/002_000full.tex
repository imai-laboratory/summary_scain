\documentclass[11pt]{amsart}
\usepackage{geometry}                % See geometry.pdf to learn the layout options. There are lots.
\geometry{a3paper}                   % ... or a4paper or a5paper or ... 
%\geometry{landscape}                % Activate for for rotated page geometry
%\usepackage[parfill]{parskip}    % Activate to begin paragraphs with an empty line rather than an indent
\usepackage{graphicx}
\usepackage{amssymb}
\usepackage{epstopdf}
\usepackage{xcolor}
\usepackage{udline}
\DeclareGraphicsRule{.tif}{png}{.png}{`convert #1 `dirname #1`/`basename #1 .tif`.png}
\renewcommand{\baselinestretch}{1.5}
\newenvironment{hangall}[1]{\hangindent = 2.5zw\everypar{\hangindent = 2.5zw}}{}

\title{}
\author{}
%\date{}                                           % Activate to display a given date or no date

\begin{document}
\maketitle
%\section{}
%\subsection{}  
\begin{hangall}{}%
A: はじめまして。こんにちは。

B: こんにちは。わたしはいま高校で陸上部に入ってるんですけど。

A: 高校生なんですね!私はエステティシャンをしています。

B: かっこいいですね。実は早起きが苦手で朝練が辛いんですよね。

A: わかりますよ。私も部活をしていた頃は、早起きが辛かったです。

B: ですよね。どうやって朝起きしていらっしゃったんですか?

A: 目覚ましを何個もかけていましたね。母親も起こしてくれましたし。

B: わたしはおばあちゃんっ子なので、おばあちゃんが起こしてくれるんですが、ソフト過ぎて起きられないんですよね。

A: なんだか想像できます。ちなみに、好きな作家は宮沢賢治ですが、読書はしますか?

B: あまり読みませんが、その本は知ってます。ファンタジーですよね。

A: \ul{そうなんですよ。ぜひ読んでみてください。}\end{hangall}
\end{document}  