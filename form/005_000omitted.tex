\documentclass[11pt]{amsart}
\usepackage{geometry}                % See geometry.pdf to learn the layout options. There are lots.
\geometry{a3paper}                   % ... or a4paper or a5paper or ... 
%\geometry{landscape}                % Activate for for rotated page geometry
%\usepackage[parfill]{parskip}    % Activate to begin paragraphs with an empty line rather than an indent
\usepackage{graphicx}
\usepackage{amssymb}
\usepackage{epstopdf}
\usepackage{xcolor}
\usepackage{udline}
\DeclareGraphicsRule{.tif}{png}{.png}{`convert #1 `dirname #1`/`basename #1 .tif`.png}
\renewcommand{\baselinestretch}{1.5}
\newenvironment{hangall}[1]{\hangindent = 2.5zw\everypar{\hangindent = 2.5zw}}{}

\title{}
\author{}
%\date{}                                           % Activate to display a given date or no date

\begin{document}
\maketitle
%\section{}
%\subsection{}  
\begin{hangall}{}%
A: あ、どうも。お元気してました?

B: はい!相変わらず、食べ歩きが好きなので、いっつも食べて歩いて元気いっぱいですよ!

A: それは良かったですね。わたしは子どもの頃から食が細くて太れない体質なので、たくさん食べれる人が羨ましいですよ。

B: そうなんですね。わたしは歩いてるからでしょうか、食べてもスタイルが良いって言われますが、太れない体質っていうのも大変そうですね。

A: ええ。とくにわたしは男なので、ひょろひょろだと、女性受けが悪いでしょう?

B: そうですね。サイクリングは足の筋肉ですもんね。わたし、セラピストをしているんですが、体型で悩んでいる方も多いですよ。

A: そうだったんですか。セラピストの方は、自分の悩みはご家族に相談することになるんですか?

B: はい。とは言っても、私の場合は、なかなか悩みってないですけどね。猫舌で熱いものが苦手なことくらいでしょうか。

A: \ul{おや、可愛らしい悩みですね!こんな話をしていたら、久しぶりに家族に会いたくなりました。旅行ばかりしていないで、たまには島根から実家へ帰省しようかな。}\end{hangall}
\end{document}  