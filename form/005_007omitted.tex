\documentclass[11pt]{amsart}
\usepackage{geometry}                % See geometry.pdf to learn the layout options. There are lots.
\geometry{a3paper}                   % ... or a4paper or a5paper or ... 
%\geometry{landscape}                % Activate for for rotated page geometry
%\usepackage[parfill]{parskip}    % Activate to begin paragraphs with an empty line rather than an indent
\usepackage{graphicx}
\usepackage{amssymb}
\usepackage{epstopdf}
\usepackage{xcolor}
\usepackage{udline}
\DeclareGraphicsRule{.tif}{png}{.png}{`convert #1 `dirname #1`/`basename #1 .tif`.png}
\renewcommand{\baselinestretch}{1.5}
\newenvironment{hangall}[1]{\hangindent = 2.5zw\everypar{\hangindent = 2.5zw}}{}

\title{}
\author{}
%\date{}                                           % Activate to display a given date or no date

\begin{document}
\maketitle
%\section{}
%\subsection{}  
\begin{hangall}{}%
A: すみません。ファックスの調子が悪いので見てもらってもいいですか?

B: えー!中学生のわたしが見ても良いの?かえって壊しちゃうかもよ?

A: わたし本当に機械に弱くて。こういうのって若い人の方が得意なんじゃないの?

B: 若い人って言ってもさあ、限度があるじゃん!まあ、見てあげてもいいよ?わたし、お母さんからも頼られてるし!

A: それは頼もしい!お母さんと仲いいのね。

B: 仲がいいっていうか、お母さんとわたし、二人っきりの家族だからね。

A: 本当だね。よかったらお友達にならない?わたし趣味で手芸をやってるんだけど一緒にやってみない?

B: やってみたい!でも、帰りが遅くなるのはちょっと困るな。暗いところが苦手だから、帰り道が怖いよ。

A: その時は送って言ってあげるから安心して。わたし普段はインドア派なんだけどそれくらいするわよ!

B: \ul{わあ、親切な人で良かった。ありがとう!}\end{hangall}
\end{document}  