\documentclass[11pt]{amsart}
\usepackage{geometry}                % See geometry.pdf to learn the layout options. There are lots.
\geometry{a3paper}                   % ... or a4paper or a5paper or ... 
%\geometry{landscape}                % Activate for for rotated page geometry
%\usepackage[parfill]{parskip}    % Activate to begin paragraphs with an empty line rather than an indent
\usepackage{graphicx}
\usepackage{amssymb}
\usepackage{epstopdf}
\usepackage{xcolor}
\usepackage{udline}
\DeclareGraphicsRule{.tif}{png}{.png}{`convert #1 `dirname #1`/`basename #1 .tif`.png}
\renewcommand{\baselinestretch}{1.5}
\newenvironment{hangall}[1]{\hangindent = 2.5zw\everypar{\hangindent = 2.5zw}}{}

\title{}
\author{}
%\date{}                                           % Activate to display a given date or no date

\begin{document}
\maketitle
%\section{}
%\subsection{}  
\begin{hangall}{}%
A: こんにちは、近頃寒いですがお元気ですか?

B: はい、おかげさまでとても元気ですよ!スポーツが好きで、観戦するのも自分でやるのも好きですね。最近はテニスがお気に入りです。

A: スポーツ好きなのですね。わたし体を動かすのがとっても苦手で。雪かきが得意って言えるくらい腕力と体力はあるんですけど。

B: 雪かきですか!寒い地域にお住まいなんですか?

A: 長野で生まれ育ちましたよ。あなたはどちらにお住まいですか?

B: わたしは都内です。マッサージ師として働いているんです。お仕事はなんですか?

A: 都会憧れます。わたしは薬剤師として働いていますよ。実はこの間誕生日だったんです。クリスマスイブ生まれなんですけど、恋人もいないので仕事三昧でした。

B: そうなんですか、少し似ているお仕事ですね。ご趣味は何かありますか?

A: 今趣味を探しているところなんです。あなたの趣味は何かありますか?

B: わたしはずっとスポーツ一筋ですね。今は観戦とテニスですが、昔はバレーボール部の主将を務めるほどでした。

A: \ul{テニスもバレーボールもできてすごいです。わたしもあなたを見習って運動を始めてみることにします!}\end{hangall}
\end{document}  