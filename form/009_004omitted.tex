\documentclass[11pt]{amsart}
\usepackage{geometry}                % See geometry.pdf to learn the layout options. There are lots.
\geometry{a3paper}                   % ... or a4paper or a5paper or ... 
%\geometry{landscape}                % Activate for for rotated page geometry
%\usepackage[parfill]{parskip}    % Activate to begin paragraphs with an empty line rather than an indent
\usepackage{graphicx}
\usepackage{amssymb}
\usepackage{epstopdf}
\usepackage{xcolor}
\usepackage{udline}
\DeclareGraphicsRule{.tif}{png}{.png}{`convert #1 `dirname #1`/`basename #1 .tif`.png}
\renewcommand{\baselinestretch}{1.5}
\newenvironment{hangall}[1]{\hangindent = 2.5zw\everypar{\hangindent = 2.5zw}}{}

\title{}
\author{}
%\date{}                                           % Activate to display a given date or no date

\begin{document}
\maketitle
%\section{}
%\subsection{}  
\begin{hangall}{}%
A: こんにちは。お元気ですか?

B: こんにちは、最近出身の埼玉から東京に引っ越しました。

A: そうなんですね。わたしは実家のある群馬に暮らしています。

B: 地元は落ち着くし、道に迷わないので、安心ですよね。地図が読めず、方向音痴なので、いつもスマホ片手に出歩いでいます。

A: そうなんですね。今は、スマホもとても便利ですよね。

B: そうですね、わたしは東京で暮らしているものの、就活中なんですが、何をされている方なんですか?

A: 就活、がんばってくださいね!わたしは音楽鑑賞が趣味なので、楽器メーカーで働いているんです。

B: いいですね、わたしも音楽大好きです。群馬の暮らしは快適ですか?

A: 音楽いいですよね。音楽の秋に生まれたので、母がよく音楽を胎教で聞かせてくれていたらしいです。群馬も、自然いっぱいでいいですよ。

B: 田舎の方が遮るものがない分、怖く感じるからかもしれません。

A: \ul{なるほど、そういうことですね。}\end{hangall}
\end{document}  