\documentclass[11pt]{amsart}
\usepackage{geometry}                % See geometry.pdf to learn the layout options. There are lots.
\geometry{a3paper}                   % ... or a4paper or a5paper or ... 
%\geometry{landscape}                % Activate for for rotated page geometry
%\usepackage[parfill]{parskip}    % Activate to begin paragraphs with an empty line rather than an indent
\usepackage{graphicx}
\usepackage{amssymb}
\usepackage{epstopdf}
\usepackage{xcolor}
\usepackage{udline}
\DeclareGraphicsRule{.tif}{png}{.png}{`convert #1 `dirname #1`/`basename #1 .tif`.png}
\renewcommand{\baselinestretch}{1.5}
\newenvironment{hangall}[1]{\hangindent = 2.5zw\everypar{\hangindent = 2.5zw}}{}

\title{}
\author{}
%\date{}                                           % Activate to display a given date or no date

\begin{document}
\maketitle
%\section{}
%\subsection{}  
\begin{hangall}{}%
A: こんにちは!昔、チアリーダーをやってたので、たまに身体を動かしたくて、ウズウズしてくることがあるんですよ。

B: へぇ。すごく活動的な方なんですね。わたしは占いに興味があるくらいで、あんまり運動の方はできないんです。

A: 占い!面白いですよね。わたしが教えてる生徒たちも、雑誌の付録ページを見ては、いつも盛り上がってますよ!

B: あはは、やっぱりそうなんだ!雑誌の付録ページは占いの定番ですもんね!ところで、教師の方だったんですか?

A: ええ、そうなんです。昔はチアなんてやって弾けてましたけど、今はすっかり大人しくなって。学校の近くに住んでるので、浮いた話もないんです、ハハハ。

B: わたしも、浮いた話なんてないですよ。猫を飼ってるんですが、家で猫とずーっと戯れてますもの。でも、大晦日だけは賑やかなんです。

A: なるほど、そういうことでしたか!毎年、楽しい年越しでいいですね。わたしの実家は秋田で離れてるので、毎年は帰れないんですよ。

B: \ul{そうなんですね。それは寂しいですよね。あ、良かったら、今年の大みそかは、わたしの家にいらっしゃいません?プレゼントなんて用意しなくて良いですから。}\end{hangall}
\end{document}  