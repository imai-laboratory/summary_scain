\documentclass[11pt]{amsart}
\usepackage{geometry}                % See geometry.pdf to learn the layout options. There are lots.
\geometry{a3paper}                   % ... or a4paper or a5paper or ... 
%\geometry{landscape}                % Activate for for rotated page geometry
%\usepackage[parfill]{parskip}    % Activate to begin paragraphs with an empty line rather than an indent
\usepackage{graphicx}
\usepackage{amssymb}
\usepackage{epstopdf}
\usepackage{xcolor}
\usepackage{udline}
\DeclareGraphicsRule{.tif}{png}{.png}{`convert #1 `dirname #1`/`basename #1 .tif`.png}
\renewcommand{\baselinestretch}{1.5}
\newenvironment{hangall}[1]{\hangindent = 2.5zw\everypar{\hangindent = 2.5zw}}{}

\title{}
\author{}
%\date{}                                           % Activate to display a given date or no date

\begin{document}
\maketitle
%\section{}
%\subsection{}  
\begin{hangall}{}%
A: こんにちは!毎日暑いですね!夏って、昔からこんなに暑かったでしたっけ?

B: ここまで昔は暑くなかったと思いますよ!ここ富山は涼しそうなイメージだったのに、いざ嫁いで来てみると暑くて大変です!

A: えー、富山の方もそうなんですか。わたしは、いまは東京住まいなんですけど、こっちはゲリラ豪雨もすごいんですよ。わたしは雷が苦手なのでほんと辛い。

B: 近年、豪雨や雷が酷いですもんね。早く夏が終わらないかな!?誕生日がひな祭りの日なので、暑いのが苦手なんです。

A: ああ、早生まれなんですね。わたしなんて、夏生まれなのに、暑いのが苦手ですよ。

B: 確かに、「暑いの得意!」って人をあまり見たことがないですね。ところで、お仕事とかってされてるんですか?

A: 大変なお仕事をされてるんですね。でも、とっても意義があるお仕事で尊敬します!もう富山に嫁がれて長いんですか?

B: いえ、まだ数年位しか経ってないです。やっと土地柄とかに色々慣れてきた感じです。

A: \ul{なるほどー。わたしなんて、方向音痴な上に地図が読めないので、いまだに東京で迷子になりますよ!}\end{hangall}
\end{document}  