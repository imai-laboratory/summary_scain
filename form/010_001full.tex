\documentclass[11pt]{amsart}
\usepackage{geometry}                % See geometry.pdf to learn the layout options. There are lots.
\geometry{a3paper}                   % ... or a4paper or a5paper or ... 
%\geometry{landscape}                % Activate for for rotated page geometry
%\usepackage[parfill]{parskip}    % Activate to begin paragraphs with an empty line rather than an indent
\usepackage{graphicx}
\usepackage{amssymb}
\usepackage{epstopdf}
\usepackage{xcolor}
\usepackage{udline}
\DeclareGraphicsRule{.tif}{png}{.png}{`convert #1 `dirname #1`/`basename #1 .tif`.png}
\renewcommand{\baselinestretch}{1.5}
\newenvironment{hangall}[1]{\hangindent = 2.5zw\everypar{\hangindent = 2.5zw}}{}

\title{}
\author{}
%\date{}                                           % Activate to display a given date or no date

\begin{document}
\maketitle
%\section{}
%\subsection{}  
\begin{hangall}{}%
A: こんにちは。わたし、ホームヘルパーをしているんですが、疲れてもやりがいがある仕事はいいですね。

B: こんにちは。ほんと偉い仕事だと思うわ。その仕事を選んだのは、ご両親の影響とかなの?

A: 両親は特に関係ないですね。わたし、力だけには自信があるので、この力を活かせる仕事に就こうかなって。ほら、見てください、この厚い胸板!

B: わあ!すごいわー。でも自分の長所をいかせる仕事に就けるって素敵よね。わたしなんて、全然関係ない仕事だもの。

A: そうなんですか?どんな仕事をなさってるんですか?

B: 事務職をしてるの。昔から手先が器用で、ほんとうは美容師になりたかったんだけど、諦めちゃった。

A: そうだったんですね。それはもったいないですね。ところで、山梨のご出身だったりしませんか?

B: ううん。別のところよ。今いる岐阜も、引っ越してきてまだ日が浅いの。

A: そうなんですかぁ。実はわたし、山梨の方言が好きなんです。だから、ネイティブな方の方言を生で聞いてみたくって。でも、なかなか出会わないんですよね。

B: \ul{確かに、山梨の方言ってあまり聞かないわね。でもイメージ的にはホッコリしてそうね!}\end{hangall}
\end{document}  