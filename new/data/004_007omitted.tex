\documentclass[11pt]{amsart}
\usepackage{geometry}                % See geometry.pdf to learn the layout options. There are lots.
\geometry{a3paper}                   % ... or a4paper or a5paper or ... 
%\geometry{landscape}                % Activate for for rotated page geometry
%\usepackage[parfill]{parskip}    % Activate to begin paragraphs with an empty line rather than an indent
\usepackage{graphicx}
\usepackage{amssymb}
\usepackage{epstopdf}
\usepackage{xcolor}
\usepackage{udline}
\DeclareGraphicsRule{.tif}{png}{.png}{`convert #1 `dirname #1`/`basename #1 .tif`.png}
\renewcommand{\baselinestretch}{1.5}
\newenvironment{hangall}[1]{\hangindent = 2.5zw\everypar{\hangindent = 2.5zw}}{}

\title{}
\author{}
%\date{}                                           % Activate to display a given date or no date

\begin{document}
\maketitle
%\section{}
%\subsection{}  
\begin{hangall}{}%
A: ねえねえ、あなたって今住んでるところで、生まれ育った?

B: 今住んでるところは違うよ。出身は和歌山なんだ。

A: そうなんだー。わたしは今神戸で、兵庫県以外で暮らしたことがないんだよねー。ほかの県にも住んだことある?

B: 住んだことはないんだけど、アニメの聖地巡礼をよくやってるから、色んな県に行くことはあるよ。

A: いいなあ!わたし、通訳って職業柄、海外旅行にはよく行くんだけど、考えてみたら、日本のことをあまり知らなくてさあ。

B: そうなんだー。でも、わたしも日本のことはよく知らないなぁ。アニメに関係してればわかるんだけどね。

A: それで十分だよ。だって、海外からしたら、アニメって日本の代名詞みたいなものだと思わない?

B: そう思われがちだよね。でも、仕事も肉体労働だし、結構体は動かす方なんだよね。

A: そうなんだね!わたしもジッとしてられないほうだから、気が合うかもね、わたしたち!

B: そうだといいなぁ。人からよく、お前マイペースだなって言われてるけど、意外と気が合うのかもね。

A: \ul{うん、絶対そう思う!}\end{hangall}
\end{document}  