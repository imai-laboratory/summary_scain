\documentclass[11pt]{amsart}
\usepackage{geometry}                % See geometry.pdf to learn the layout options. There are lots.
\geometry{a3paper}                   % ... or a4paper or a5paper or ... 
%\geometry{landscape}                % Activate for for rotated page geometry
%\usepackage[parfill]{parskip}    % Activate to begin paragraphs with an empty line rather than an indent
\usepackage{graphicx}
\usepackage{amssymb}
\usepackage{epstopdf}
\usepackage{xcolor}
\usepackage{udline}
\DeclareGraphicsRule{.tif}{png}{.png}{`convert #1 `dirname #1`/`basename #1 .tif`.png}
\renewcommand{\baselinestretch}{1.5}
\newenvironment{hangall}[1]{\hangindent = 2.5zw\everypar{\hangindent = 2.5zw}}{}

\title{}
\author{}
%\date{}                                           % Activate to display a given date or no date

\begin{document}
\maketitle
%\section{}
%\subsection{}  
\begin{hangall}{}%
A: ああ、もう!ダンスの練習に行ったら、友達がみんな遅刻してきたの!参っちゃう!

B: わかるわー。わたしも以前チアリーダーをやってたから。そういうときって本当に困るわよね。

A: 分かってくれて嬉しい!ただ、前に沖縄にいたときは、皆、もっともーっと時間にルーズだったけどね。

B: そうだったの?わたしは秋田で生まれたけど、秋田の人たちはそんなことなかったわ。住んでるところによって、性格って変わってくるのね。

A: そうかも!わたしの母は、時間にとっても厳しくてね。わたし、そんな母のことをとっても尊敬してるわ。

B: 良いことね。わたし、教師をやってるんだけど、学校のそばに住んでるのよ。遅刻してくる子たちを何人も見たわ。親がしっかりしてると違うのにね。

A: へえ、教師をやってるなんて、尊敬しちゃう。子どもって、時々イラッとさせられない?わたしって短気だから、教師は絶対ムリ!

B: そうよね。若いものね。わたしは70年代生まれよ。まぁまだまだ若いと自分では思ってるんだけどね。

A: へえー!70年代って言ったら、フォークとか流行ってた頃よね。色んな時代を見てきたからこそ、落ち着いて子どもに教えられるのかもね。

B: \ul{そう言ってもらえると、自分に自信が出てくるわ。ありがとう。}\end{hangall}
\end{document}  