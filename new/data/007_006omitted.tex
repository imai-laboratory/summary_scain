\documentclass[11pt]{amsart}
\usepackage{geometry}                % See geometry.pdf to learn the layout options. There are lots.
\geometry{a3paper}                   % ... or a4paper or a5paper or ... 
%\geometry{landscape}                % Activate for for rotated page geometry
%\usepackage[parfill]{parskip}    % Activate to begin paragraphs with an empty line rather than an indent
\usepackage{graphicx}
\usepackage{amssymb}
\usepackage{epstopdf}
\usepackage{xcolor}
\usepackage{udline}
\DeclareGraphicsRule{.tif}{png}{.png}{`convert #1 `dirname #1`/`basename #1 .tif`.png}
\renewcommand{\baselinestretch}{1.5}
\newenvironment{hangall}[1]{\hangindent = 2.5zw\everypar{\hangindent = 2.5zw}}{}

\title{}
\author{}
%\date{}                                           % Activate to display a given date or no date

\begin{document}
\maketitle
%\section{}
%\subsection{}  
\begin{hangall}{}%
A: ねえ、聞いて。今日もまた学校の先生に怒られちゃったよ。

B: ええ?どうしたの?何があったの?

A: 今日さ、看護学校の実習で、注射打たなきゃいけなくて、失敗しまくり。注射ってホント苦手なんだよね。

B: そうだったの。わかるわ。わたしも、看護学校に通ってた時は注射苦手だったもの。今はもう助産師になって長いから、慣れっこだけどね。

A: へえ!助産師さんなんだ!ねえ、血とか怖くない?わたしの学校、千葉にあるんだけど、そこの友達は血が怖いから別の進路に変更したんだよ。

B: もう怖くないわ。だって、血を怖がってたら、助産師の仕事なんて、全然務まらないもの。

A: そうだよねー。わたしも頑張らなくちゃなー。なんかコツとかあるのかな?

B: そうねぇ、慣れかしら。あとは、頑張った自分にご褒美をあげたりね。わたしの場合は趣味のカラオケ。

A: イベントかあ。それなら、わたしの誕生日はバレンタインデーなんだけど、月命日ならぬ月誕生日に、ゴディバのチョコを自分にプレゼントしようかな。

B: あ!それも良いんじゃない!?そうすれば、きっと、みるみるうちに、看護の実習も上達するわよ!

A: \ul{そうなるといいなー!いいアドバイスをありがとう!}\end{hangall}
\end{document}  