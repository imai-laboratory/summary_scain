\documentclass[11pt]{amsart}
\usepackage{geometry}                % See geometry.pdf to learn the layout options. There are lots.
\geometry{a3paper}                   % ... or a4paper or a5paper or ... 
%\geometry{landscape}                % Activate for for rotated page geometry
%\usepackage[parfill]{parskip}    % Activate to begin paragraphs with an empty line rather than an indent
\usepackage{graphicx}
\usepackage{amssymb}
\usepackage{epstopdf}
\usepackage{xcolor}
\usepackage{udline}
\DeclareGraphicsRule{.tif}{png}{.png}{`convert #1 `dirname #1`/`basename #1 .tif`.png}
\renewcommand{\baselinestretch}{1.5}
\newenvironment{hangall}[1]{\hangindent = 2.5zw\everypar{\hangindent = 2.5zw}}{}

\title{}
\author{}
%\date{}                                           % Activate to display a given date or no date

\begin{document}
\maketitle
%\section{}
%\subsection{}  
\begin{hangall}{}%
A: こんにちは。わたし、本を読むのが趣味なんだけど、あなたは趣味ってある?

B: こんにちは!わたしはね、古本屋によく行くほうかなあ。本屋って独特の匂いがすると思わない?

A: するする!わかる!わたしもね、翻訳家の仕事柄、古本屋にも行かないといけない時があって、よく行くのよ。

B: そうなんだ!中には、本屋に行くとトイレに行きたくなるって人もいるけど、あなたもそのタイプ?

A: いるよね、そういうタイプ!わたしは大丈夫よ!

B: わたしも大丈夫!っていうか、むしろ、あの匂いが懐かしく感じるのよねー。学校の教科書とかプリントを思い出すっていうか。

A: \ul{そうね、たしかに学校のものも、そんな匂いがしてたっけ。学校では図書館に籠って、よくテスト勉強の暗記をしてたわ。わたし、暗記は得意なのよ。}\end{hangall}
\end{document}  