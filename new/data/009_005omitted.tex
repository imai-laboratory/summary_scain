\documentclass[11pt]{amsart}
\usepackage{geometry}                % See geometry.pdf to learn the layout options. There are lots.
\geometry{a3paper}                   % ... or a4paper or a5paper or ... 
%\geometry{landscape}                % Activate for for rotated page geometry
%\usepackage[parfill]{parskip}    % Activate to begin paragraphs with an empty line rather than an indent
\usepackage{graphicx}
\usepackage{amssymb}
\usepackage{epstopdf}
\usepackage{xcolor}
\usepackage{udline}
\DeclareGraphicsRule{.tif}{png}{.png}{`convert #1 `dirname #1`/`basename #1 .tif`.png}
\renewcommand{\baselinestretch}{1.5}
\newenvironment{hangall}[1]{\hangindent = 2.5zw\everypar{\hangindent = 2.5zw}}{}

\title{}
\author{}
%\date{}                                           % Activate to display a given date or no date

\begin{document}
\maketitle
%\section{}
%\subsection{}  
\begin{hangall}{}%
A: こんにちは、最近趣味のネットサーフィンにはまってるんですが、何かハマっているものはありますか?

B: こんにちは。わたしは、音楽鑑賞が趣味なので、毎日音楽を聴いています。

A: 音楽鑑賞も楽しいですよね!わたしはドラックストアの店員なんですが、遅刻魔なもので、よく遅刻しては怒られてます。

B: 音楽好きが極まって、楽器メーカーに勤務しています。ドラッグストア、お仕事大変そうなので寝坊するのもわかる気がします。

A: すごい!楽器に精通していそうですね。音楽以外では何か他に趣味とかあるんですか?

B: 他には趣味はないんですが、秋の紅葉が好きでよく見に行きます。夕暮れ時が特に懐かしく感じられてよいんですよ。

A: へえ、自然鑑賞もまた風流な感じがしますね。わたしは栃木に実家があるんですが、栃木の秋の紅葉は美しいですよ。

B: 栃木の紅葉、見に行ったことがあります!実は、群馬出身なんですよ。

A: そうですね、ぜひご一緒したいものです。

B: \ul{また紅葉の頃、お休みの日をお知らせしますね。}\end{hangall}
\end{document}  