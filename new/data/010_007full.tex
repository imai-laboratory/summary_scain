\documentclass[11pt]{amsart}
\usepackage{geometry}                % See geometry.pdf to learn the layout options. There are lots.
\geometry{a3paper}                   % ... or a4paper or a5paper or ... 
%\geometry{landscape}                % Activate for for rotated page geometry
%\usepackage[parfill]{parskip}    % Activate to begin paragraphs with an empty line rather than an indent
\usepackage{graphicx}
\usepackage{amssymb}
\usepackage{epstopdf}
\usepackage{xcolor}
\usepackage{udline}
\DeclareGraphicsRule{.tif}{png}{.png}{`convert #1 `dirname #1`/`basename #1 .tif`.png}
\renewcommand{\baselinestretch}{1.5}
\newenvironment{hangall}[1]{\hangindent = 2.5zw\everypar{\hangindent = 2.5zw}}{}

\title{}
\author{}
%\date{}                                           % Activate to display a given date or no date

\begin{document}
\maketitle
%\section{}
%\subsection{}  
\begin{hangall}{}%
A: こんにちは。わたし、京都で大学院生をしています。あなたはどんな方ですか?

B: こんにちは!私は、普通の会社で普通のOLをしてるよ。大学院に行くなんて、よっぽど頭が良いんだね!

A: 頭はそんなに良くないですよ。空気を読むことが苦手なので、本当は就職した方が良かったんだろうけど、そのまま進学してしまったんです。

B: いわゆるKYってやつ?そんなこと言うのって、周りの僻みだから無視したほうが良いよ。私、そういう悪口言う人って大嫌い。

A: 本当ですか?そういってもらえると、すごく安心します。わたし、グラマーな体型なので、悪口を言われることも多いんです。

B: あー、スタイルも良いなら、完璧、僻みだって!私も、人よりウエストが細いってだけで、会社で悪口言われたりするもん。

A: そういう悩み持ってる方って多いんですかねぇ。悪口言われる方の気にもなってほしいです。

B: 同感だよー。わたしの場合、地域性もあるかもしれない。今住んでる所って、東北で結構、閉鎖的だからさ。

A: \ul{そうなんですね。東北の方は、私行ったことがないので、今度行ってみたいです。}\end{hangall}
\end{document}  