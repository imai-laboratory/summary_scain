\documentclass[11pt]{amsart}
\usepackage{geometry}                % See geometry.pdf to learn the layout options. There are lots.
\geometry{a3paper}                   % ... or a4paper or a5paper or ... 
%\geometry{landscape}                % Activate for for rotated page geometry
%\usepackage[parfill]{parskip}    % Activate to begin paragraphs with an empty line rather than an indent
\usepackage{graphicx}
\usepackage{amssymb}
\usepackage{epstopdf}
\usepackage{xcolor}
\usepackage{udline}
\DeclareGraphicsRule{.tif}{png}{.png}{`convert #1 `dirname #1`/`basename #1 .tif`.png}
\renewcommand{\baselinestretch}{1.5}
\newenvironment{hangall}[1]{\hangindent = 2.5zw\everypar{\hangindent = 2.5zw}}{}

\title{}
\author{}
%\date{}                                           % Activate to display a given date or no date

\begin{document}
\maketitle
%\section{}
%\subsection{}  
\begin{hangall}{}%
A: こんにちは。お話して大丈夫ですか?

B: はい、大丈夫です。今、飼っている熱帯魚の世話をしていたところです。

A: ほほお熱帯魚ですか。きれいなことでしょうね。

B: はい、私はエステティシャンをしていて、普段忙しいのでとても癒されます。

A: 華やかそうなお仕事で羨ましいですね。わたしは先日定年を迎えて自由の身になりました。

B: そうなんですね。ゆっくりされてくださいね。私はネイリストも目指しているところです。

A: それでは総合美容家を目指せそうですね。わたしは体で自慢できるのは虫歯がないことぐらいなんで、美しさに磨きをかける方は尊敬します。

B: ありがとうございます。虫歯がないなんて、とても羨ましいです!

A: その代わり髪は薄いので、美容とは縁がありません。

B: 定年されたので、これからおしゃれしても良いと思いますよ。ちなみに私は岩手出身ですが、あなたはどちらですか?

A: 岩手の方なんですか。わたしは宮城の出身なので近いですね。

B: \ul{それは近いですね!奇遇です。}\end{hangall}
\end{document}  