\documentclass[11pt]{amsart}
\usepackage{geometry}                % See geometry.pdf to learn the layout options. There are lots.
\geometry{a3paper}                   % ... or a4paper or a5paper or ... 
%\geometry{landscape}                % Activate for for rotated page geometry
%\usepackage[parfill]{parskip}    % Activate to begin paragraphs with an empty line rather than an indent
\usepackage{graphicx}
\usepackage{amssymb}
\usepackage{epstopdf}
\usepackage{xcolor}
\usepackage{udline}
\DeclareGraphicsRule{.tif}{png}{.png}{`convert #1 `dirname #1`/`basename #1 .tif`.png}
\renewcommand{\baselinestretch}{1.5}
\newenvironment{hangall}[1]{\hangindent = 2.5zw\everypar{\hangindent = 2.5zw}}{}

\title{}
\author{}
%\date{}                                           % Activate to display a given date or no date

\begin{document}
\maketitle
%\section{}
%\subsection{}  
\begin{hangall}{}%
A: あら、こんにちは。今から、スーパーのパートに行くところなんですよー。

B: こんにちは。そうなんですね。わたしは職場の楽器メーカーに出勤するところです。

A: お勤めご苦労さま。楽器メーカーにお勤めってことは、ご自身もなにか楽器をやるの?

B: いえ、演奏はせず、聴く方専門ですね。音楽鑑賞が趣味なんです。

A: 素敵な趣味ねえ。わたしなんて、貯金が趣味だから、そんな優雅な時間を過ごしたことないわ!

B: そうなんですか?でも、貯金も素敵な趣味だと思いますよー。だって、わたしなんて、音楽の方にお金つぎ込んじゃってますもん。

A: それはそれで、良いお金の使い方だと思うわ。それじゃ、休みの日は、音楽を聞いて過ごす感じなの?

B: はい。特に休みの日の夕暮れ時。あの時間に音楽を聴いていると、とっても懐かしい気分になるんです。

A: そうね、わたしもこの勉強嫌いの硬い頭を叩いて、曲名を思い出す努力をしてみるわ!

B: \ul{仕事終わりが楽しみになってきました!お話しできて良かったです!}\end{hangall}
\end{document}  