\documentclass[11pt]{amsart}
\usepackage{geometry}                % See geometry.pdf to learn the layout options. There are lots.
\geometry{a3paper}                   % ... or a4paper or a5paper or ... 
%\geometry{landscape}                % Activate for for rotated page geometry
%\usepackage[parfill]{parskip}    % Activate to begin paragraphs with an empty line rather than an indent
\usepackage{graphicx}
\usepackage{amssymb}
\usepackage{epstopdf}
\usepackage{xcolor}
\usepackage{udline}
\DeclareGraphicsRule{.tif}{png}{.png}{`convert #1 `dirname #1`/`basename #1 .tif`.png}
\renewcommand{\baselinestretch}{1.5}
\newenvironment{hangall}[1]{\hangindent = 2.5zw\everypar{\hangindent = 2.5zw}}{}

\title{}
\author{}
%\date{}                                           % Activate to display a given date or no date

\begin{document}
\maketitle
%\section{}
%\subsection{}  
\begin{hangall}{}%
A: いやー、今年の冬は雪が多いですね!東京人は雪に慣れてないから、ちょっと降っただけでも、転ぶ人が多いとか。

B: そうなんですよ。わたしは滋賀県で開業した病院で医師をやってるんですが、東京の医師仲間は、そういう転倒患者さんをよく看ているらしいです。

A: あー、やっぱりねー!長野で生まれ育ったんですが、雪道の歩き方にコツがいるんですよ。あなたは滋賀で生まれ育ったんですか?

B: はい、滋賀で生まれ育ちました。車の運転が下手なので、こちらには今日は新幹線で来ています。

A: そうだったんですか、ご苦労なことです。今朝は長野でも大雪が降りましてね、わたしは朝から雪かきをしとりました。

B: そうだったんですか!大雪の中の雪かきは大変ですよね。苦労はお互い様ですね。わたし、ゴルフが得意なんですが、この時期は雪でゴルフができないのが残念です。

A: ああ、雪解けが待ち遠しいですねえ。雪かきはかなり得意な方なんですが、今朝は熱中しすぎて仕事に遅刻しそうになっちゃいましたよ。

B: そうですよね。医者がそんなことで遅刻したら、信用問題です。もう、それからは朝はコーヒーと固定して飲むことにしました。

A: なるほど、それは名案です。しかし、明日は筋肉痛になりそうだな。疲労にはレモンが良いと聞くが、酸っぱいものが苦手なんです。どうしましょ?

B: そうですねぇ。牡蠣やしじみなどにも、疲労回復、滋養効果がありますので、そういうものなら酸っぱくなくて良いかもしれませんよ。

A: \ul{なるほどなるほど!さすがお医者さんですな!良いことを聞きました!}\end{hangall}
\end{document}  