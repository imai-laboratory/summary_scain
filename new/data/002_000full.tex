\documentclass[11pt]{amsart}
\usepackage{geometry}                % See geometry.pdf to learn the layout options. There are lots.
\geometry{a3paper}                   % ... or a4paper or a5paper or ... 
%\geometry{landscape}                % Activate for for rotated page geometry
%\usepackage[parfill]{parskip}    % Activate to begin paragraphs with an empty line rather than an indent
\usepackage{graphicx}
\usepackage{amssymb}
\usepackage{epstopdf}
\usepackage{xcolor}
\usepackage{udline}
\DeclareGraphicsRule{.tif}{png}{.png}{`convert #1 `dirname #1`/`basename #1 .tif`.png}
\renewcommand{\baselinestretch}{1.5}
\newenvironment{hangall}[1]{\hangindent = 2.5zw\everypar{\hangindent = 2.5zw}}{}

\title{}
\author{}
%\date{}                                           % Activate to display a given date or no date

\begin{document}
\maketitle
%\section{}
%\subsection{}  
\begin{hangall}{}%
A: こんにちは。また会いましたね。

B: あら、ほんとですね!お住まいが近くなのかしら?

A: ええ。このスーパーの近くのアパートに住んでるんですよ。あなたもこの辺りにお住まいなんですか?

B: ええ。わたしは少し先のコンビニの近くです。お互い、お店が近いと、普段便利ですよね!

A: そうなんですよね。わたし外資系企業に勤めてるんですが、残業が多くて料理する気になれないから助かってます。

B: すごく分かります!わたしが勤めてる会社もIT系なので、激務で嫌になっちゃいます!

A: お互い忙しいですよね。あなたは自炊されてますか?

B: それがねえ、お恥ずかしながら、大体いつも、コンビニ弁当です。あなたは?

A: そこも同じですね。久しぶりに母のカツ丼が食べたいなあ。わたしにとってのお袋の味なんですよ。あなたにもありますか?そういうの。

B: うーん、わたしは母も忙しい人だったので、あまりお袋の味って思いつきません。そのかわり、甘いものには目がないですよ!

A: \ul{そうなんですね!今度あなたのおすすめを教えてください。}\end{hangall}
\end{document}  