\documentclass[11pt]{amsart}
\usepackage{geometry}                % See geometry.pdf to learn the layout options. There are lots.
\geometry{a3paper}                   % ... or a4paper or a5paper or ... 
%\geometry{landscape}                % Activate for for rotated page geometry
%\usepackage[parfill]{parskip}    % Activate to begin paragraphs with an empty line rather than an indent
\usepackage{graphicx}
\usepackage{amssymb}
\usepackage{epstopdf}
\usepackage{xcolor}
\usepackage{udline}
\DeclareGraphicsRule{.tif}{png}{.png}{`convert #1 `dirname #1`/`basename #1 .tif`.png}
\renewcommand{\baselinestretch}{1.5}
\newenvironment{hangall}[1]{\hangindent = 2.5zw\everypar{\hangindent = 2.5zw}}{}

\title{}
\author{}
%\date{}                                           % Activate to display a given date or no date

\begin{document}
\maketitle
%\section{}
%\subsection{}  
\begin{hangall}{}%
A: こんにちは、最近困っていることがあるんです、聞いてくれませんか?

B: そうなんですね。どうしたんですか?

A: 私はセロリが大嫌いなんだ、でも蛇年の私はセロリがラッキーフードなんだ。

B: なんてことでしょう!でもラッキーを手に入れるためには、その苦手な物を克服しないとですね。

A: そうなんですよ…あなたは苦手なものがありますか?

B: 私は虫が苦手なんです。でも、キャンプに興味があるので、がんばって虫を克服しようかと思っています。

A: なるほど、高所恐怖症の私にはとても危険で無理そうなことだなぁ。

B: 大丈夫ですよ。平地でも、キャンプできるところはありますので。

A: \ul{そうなんですか!今度ぜひ教えてほしいです!}\end{hangall}
\end{document}  