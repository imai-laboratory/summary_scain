\documentclass[11pt]{amsart}
\usepackage{geometry}                % See geometry.pdf to learn the layout options. There are lots.
\geometry{a3paper}                   % ... or a4paper or a5paper or ... 
%\geometry{landscape}                % Activate for for rotated page geometry
%\usepackage[parfill]{parskip}    % Activate to begin paragraphs with an empty line rather than an indent
\usepackage{graphicx}
\usepackage{amssymb}
\usepackage{epstopdf}
\usepackage{xcolor}
\usepackage{udline}
\DeclareGraphicsRule{.tif}{png}{.png}{`convert #1 `dirname #1`/`basename #1 .tif`.png}
\renewcommand{\baselinestretch}{1.5}
\newenvironment{hangall}[1]{\hangindent = 2.5zw\everypar{\hangindent = 2.5zw}}{}

\title{}
\author{}
%\date{}                                           % Activate to display a given date or no date

\begin{document}
\maketitle
%\section{}
%\subsection{}  
\begin{hangall}{}%
A: こんにちは!お元気でしたか?

B: ああ、久しぶりです。こないだ成人しましたよ。2000年ミレニアム生まれなもんで。

A: おめでとうございます!2000年と言うことは平成生まれですよね。私と一緒だ!

B: そうですよね。令和になっちゃったものなあ。急に老け込んだ感じがしますよ。運動とかされてますか?

A: わたしは柔道をやってます。警察官なもので必須なんですよね。あなたは運動は?

B: あそうか、お巡りさんって武道が必須なんでしたよね。私は高校の頃卓球部でしたが、今でもその頃の仲間と会って「温泉卓球」で遊んでます。皆温泉が好きなんで。

A: 温泉で卓球!いいですねー。わたしも非番の日に子ども2人と行きたいな〜。

B: あ、私てっきり独身だと思ってましたよ。お子さんもいて、二人も。ああ、知らなかった。

A: あはは!結構時間がかかりそうですね!頑張ってくださいね!

B: \ul{オラオラ!8回生のお通りはい、道を開けんかい! なんつってね。}\end{hangall}
\end{document}  