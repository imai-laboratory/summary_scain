\documentclass[11pt]{amsart}
\usepackage{geometry}                % See geometry.pdf to learn the layout options. There are lots.
\geometry{a3paper}                   % ... or a4paper or a5paper or ... 
%\geometry{landscape}                % Activate for for rotated page geometry
%\usepackage[parfill]{parskip}    % Activate to begin paragraphs with an empty line rather than an indent
\usepackage{graphicx}
\usepackage{amssymb}
\usepackage{epstopdf}
\usepackage{xcolor}
\usepackage{udline}
\DeclareGraphicsRule{.tif}{png}{.png}{`convert #1 `dirname #1`/`basename #1 .tif`.png}
\renewcommand{\baselinestretch}{1.5}
\newenvironment{hangall}[1]{\hangindent = 2.5zw\everypar{\hangindent = 2.5zw}}{}

\title{}
\author{}
%\date{}                                           % Activate to display a given date or no date

\begin{document}
\maketitle
%\section{}
%\subsection{}  
\begin{hangall}{}%
A: こんにちは。わたしは横浜出身で、今は救急救命士をしています。あなたは?

B: 救命士さんなんですね!わたしは、助産師をしています。

A: 助産師さんですか。お忙しいでしょう。

B: たぶん、あなたのほうが忙しいと思います。最近、お産も少ないですから。

A: そうですね、でも慣れました。疲れた時って甘いものを食べる人が多いって言いますけど、何を食べますか?わたしは甘いものが苦手で、何がいいか模索中。

B: 甘いもの、わたしは好きですが、気分転換には好きなカラオケに行きます。

A: はい、今度行ってみようかな。今彼女募集中なので、はやりの1人カラオケというやつで。

B: ヒトカラ、流行ってますし十分楽しめますよ。ちなみに、私はひな祭りの日に生まれたんです。

A: \ul{そうなんですね。お雛祭りの時だなんて、なんかいいですね。何より忘れない。}\end{hangall}
\end{document}  