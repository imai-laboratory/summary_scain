\documentclass[11pt]{amsart}
\usepackage{geometry}                % See geometry.pdf to learn the layout options. There are lots.
\geometry{a3paper}                   % ... or a4paper or a5paper or ... 
%\geometry{landscape}                % Activate for for rotated page geometry
%\usepackage[parfill]{parskip}    % Activate to begin paragraphs with an empty line rather than an indent
\usepackage{graphicx}
\usepackage{amssymb}
\usepackage{epstopdf}
\usepackage{xcolor}
\usepackage{udline}
\DeclareGraphicsRule{.tif}{png}{.png}{`convert #1 `dirname #1`/`basename #1 .tif`.png}
\renewcommand{\baselinestretch}{1.5}
\newenvironment{hangall}[1]{\hangindent = 2.5zw\everypar{\hangindent = 2.5zw}}{}

\title{}
\author{}
%\date{}                                           % Activate to display a given date or no date

\begin{document}
\maketitle
%\section{}
%\subsection{}  
\begin{hangall}{}%
A: ねえねえ!最近さー、また一段と体重が増えちゃったんだよね。

B: そうなんだ?わたし自分のことにも他の人のことにも鈍感だからわからなかったよ。

A: うーん。実は最近、朝昼晩、うどんを食べてるから、そのせいだと思うんだ。

B: そんなに好きなの!?まあわたしも三食ラーメンでもいいくらいラーメン好きだから気持ちわからなくもないけどさ。

A: あなたはラーメン派なんだね。わたしのうどん好きは、生まれつきだよ。なにせ、うどんの国、四国で生まれ育ったんだもん!

B: やっぱり生まれたところと好物って関連するよね。わたしも生まれも今住んでるとこも九州だから!ラーメンは豚骨にかぎる!

A: ああ、そういうことかあ。豚骨ラーメンって、髪の毛にも良さそうだよね。ツヤツヤになりそうだと思わない?

B: それはどうかな。髪の毛痛んでるの?

A: \ul{そうなんだ!じゃあさ、豚骨ラーメンを1日3食食べて、髪の毛に効果があるか試してみてよ!}\end{hangall}
\end{document}  