\documentclass[11pt]{amsart}
\usepackage{geometry}                % See geometry.pdf to learn the layout options. There are lots.
\geometry{a3paper}                   % ... or a4paper or a5paper or ... 
%\geometry{landscape}                % Activate for for rotated page geometry
%\usepackage[parfill]{parskip}    % Activate to begin paragraphs with an empty line rather than an indent
\usepackage{graphicx}
\usepackage{amssymb}
\usepackage{epstopdf}
\usepackage{xcolor}
\usepackage{udline}
\DeclareGraphicsRule{.tif}{png}{.png}{`convert #1 `dirname #1`/`basename #1 .tif`.png}
\renewcommand{\baselinestretch}{1.5}
\newenvironment{hangall}[1]{\hangindent = 2.5zw\everypar{\hangindent = 2.5zw}}{}

\title{}
\author{}
%\date{}                                           % Activate to display a given date or no date

\begin{document}
\maketitle
%\section{}
%\subsection{}  
\begin{hangall}{}%
A: おはよう!今日はあいにくの雨だね。まぁインドア派のわたしには、影響ないけどさ。

B: おはよ。わたし、今日ボルダリングに行くつもりだったのになあ。雨の日って湿気があるから、やりづらいんだよねー。

A: そうなんだ。雨の日って、ボルダリングには向かないんだね!知らなかったよ。

B: んー、どうだろ?人によるかも。わたしの場合、身体が柔らかいからボルダリングができてるだけで、上手な人は違うかも!

A: なるほどね。確かに、身体能力とか才能って人それぞれだもんね!わたしも、機械に弱いし。

B: そうなの、そうなの!なんとか身体の能力を最大限に引き出したくて、頑張って筋トレしてるんだよ!

A: すごい!努力家なんだね。わたし、転勤族の家庭で育ってるから、続けて何かをやるっていう癖がなくて。いつも途中で投げ出しちゃうの。

B: そういう事あるよ。続かないものは苦手なものって考えたらどう?わたしも、健康に良いっていうビタミン剤を飲んでみたけど、苦くて続かなかったもん!

A: なるほど、たしかに、好きな手芸は趣味として続いてるし。あきらめも肝心ってことだね!

B: うんうん、石の上にも三年って言うしね!お互い頑張ってこ!

A: \ul{うん!}\end{hangall}
\end{document}  