\documentclass[11pt]{amsart}
\usepackage{geometry}                % See geometry.pdf to learn the layout options. There are lots.
\geometry{a3paper}                   % ... or a4paper or a5paper or ... 
%\geometry{landscape}                % Activate for for rotated page geometry
%\usepackage[parfill]{parskip}    % Activate to begin paragraphs with an empty line rather than an indent
\usepackage{graphicx}
\usepackage{amssymb}
\usepackage{epstopdf}
\usepackage{xcolor}
\usepackage{udline}
\DeclareGraphicsRule{.tif}{png}{.png}{`convert #1 `dirname #1`/`basename #1 .tif`.png}
\renewcommand{\baselinestretch}{1.5}
\newenvironment{hangall}[1]{\hangindent = 2.5zw\everypar{\hangindent = 2.5zw}}{}

\title{}
\author{}
%\date{}                                           % Activate to display a given date or no date

\begin{document}
\maketitle
%\section{}
%\subsection{}  
\begin{hangall}{}%
A: あなたって高校のとき何か部に所属してた?

B: ううん。帰宅部だったよ。あなたは?

A: わたしは漫研に入ってたんだ。あなたは漫画とか読む?

B: 漫画は、新聞についてる4コマ漫画が好き!新聞配達員をしてるから、毎日読み放題!

A: それはいいね。4コマ漫画の面白さを知ってるあなたは上級者だよ。わたし実は先日、ゲーム制作会社に就職が決まったんだ。ゲームはやる?

B: あー、ゲームはあんまりやらないかな。早寝早起きがモットーだから、夜ご飯食べたら、すぐに寝ちゃうし。

A: 規則正しい生活をしてるんだね。わたしは早寝早起きのやり方なんて忘れちゃったよ。高校の時といえば、放課後よく石焼き芋を売る屋台が通ってたんだよね。

B: あー、石焼い芋って美味しいよねー!その屋台で、よく買ってたの?

A: そうなんだよ。大好きでさ。今でもあの石焼き芋の歌を聞くと懐かしく感じるんだよね。あなたにはそういうのない?

B: あるある!学生時代は海の近くの街に住んでて、通学の時に海沿いの道を自転車で通ってたから、今でも海の匂いを嗅ぐと、当時が蘇るよ!

A: \ul{わたしと違ってなんだかかっこいい思い出だね。アニメオタクには眩しいよ。}\end{hangall}
\end{document}  