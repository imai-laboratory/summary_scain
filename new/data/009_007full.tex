\documentclass[11pt]{amsart}
\usepackage{geometry}                % See geometry.pdf to learn the layout options. There are lots.
\geometry{a3paper}                   % ... or a4paper or a5paper or ... 
%\geometry{landscape}                % Activate for for rotated page geometry
%\usepackage[parfill]{parskip}    % Activate to begin paragraphs with an empty line rather than an indent
\usepackage{graphicx}
\usepackage{amssymb}
\usepackage{epstopdf}
\usepackage{xcolor}
\usepackage{udline}
\DeclareGraphicsRule{.tif}{png}{.png}{`convert #1 `dirname #1`/`basename #1 .tif`.png}
\renewcommand{\baselinestretch}{1.5}
\newenvironment{hangall}[1]{\hangindent = 2.5zw\everypar{\hangindent = 2.5zw}}{}

\title{}
\author{}
%\date{}                                           % Activate to display a given date or no date

\begin{document}
\maketitle
%\section{}
%\subsection{}  
\begin{hangall}{}%
A: 最悪ー!ちょっと聞いてよ、嫌なもの見ちゃった!

B: どしたの?

A: 中学校のクラスの子がさ、面白い映画あるよってDVD貸してくれたんだけど、ホラーだったの!見るんじゃなかった!

B: ホラー大嫌いなのにかわいそう。わたしが趣味の筋トレで鍛えた体で一発お見舞いしてあげようか?

A: して!わたし、お母さんと二人暮らしだから、基本的に暗くなるまで1人で留守番なの。だから暗い場所も、すっごく苦手なのにー!

B: そんな時に限って思い出したりしちゃうのよね。わたしも苦いものが苦手だから大学時代罰ゲームで青汁飲んだ時は最悪だった。

A: \ul{うわ!それ、最悪だね!やっぱり、今回のDVDも罰ゲームみたいだと思う?これってイジメ?}\end{hangall}
\end{document}  