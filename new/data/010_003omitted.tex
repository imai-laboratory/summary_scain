\documentclass[11pt]{amsart}
\usepackage{geometry}                % See geometry.pdf to learn the layout options. There are lots.
\geometry{a3paper}                   % ... or a4paper or a5paper or ... 
%\geometry{landscape}                % Activate for for rotated page geometry
%\usepackage[parfill]{parskip}    % Activate to begin paragraphs with an empty line rather than an indent
\usepackage{graphicx}
\usepackage{amssymb}
\usepackage{epstopdf}
\usepackage{xcolor}
\usepackage{udline}
\DeclareGraphicsRule{.tif}{png}{.png}{`convert #1 `dirname #1`/`basename #1 .tif`.png}
\renewcommand{\baselinestretch}{1.5}
\newenvironment{hangall}[1]{\hangindent = 2.5zw\everypar{\hangindent = 2.5zw}}{}

\title{}
\author{}
%\date{}                                           % Activate to display a given date or no date

\begin{document}
\maketitle
%\section{}
%\subsection{}  
\begin{hangall}{}%
A: あのね、聞いて!ずーっと夢だったテレビ業界に潜り込むことに成功したんだ!

B: 潜り込むって。でもよかったじゃない!おめでとう!実はわたしも夢が叶ったの!

A: え、なになに?どんな夢?

B: わたし昔からアニメオタクだったけどそれを仕事にできるんだ!ゲーム制作会社に就職が決まったんだ!

A: わお、すごいじゃん!努力したんだろうね!

B: 好きだから努力っていう風には思わないけどでも頑張ったかな。あなたも家族の方喜んだでしょう?

A: うん、妹から就職祝いにスーツもらっちゃった。うちの妹はアニメ好きだから、なんだか、あなたにシンパシーを感じるわ!

B: いい妹さんだね!わたしも一人っ子だから両親が大喜びでさ。

A: そりゃ、そうでしょう!就職の次に、ご両親が喜ぶことと言ったら、結婚して孫の顔を見せることかな?

B: \ul{そんなことないよ。そういうの好きな人もいると思うけどな。}\end{hangall}
\end{document}  